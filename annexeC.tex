\Chapter{Analyse linéaire de TOKAM anisotherme}
\label{AnnexeC}
\begin{refsection}

\section*{Analyse linéaire de TOKAM anisotherme}
\begin{align}
\label{2-eqContinuiteTemp}
\partial_t \text{N}
=& - \nabla\cdot\left(\text{N}\mathbf U_\text{E}\right) -\sigma
\text{N}\sqrt{\text{T}_e}e^{\Lambda-\Delta\Phi/\text{T}_e} + D\nabla^2 \text{N}
+ \mathcal{S}
\\[0.5cm]
\label{2-eqCourantTemp}
\begin{split}
\partial_{t}\text{W} =& 
-\nabla\cdot\left(\text{W}\mathbf U_\text{E}\right)
+\text{B}^2\sigma\sqrt{\text{T}_e}\left(1-e^{\Lambda-\Delta\Phi/\text{T}_e}\right)\\
&-\text{B}^2\nabla\cdot\left(\text{N}\mathbf
U_{\nabla\text{B}}\right)/\text{N} +\nu\nabla^2\text{W}
\end{split}
\\[0.5cm]
\label{2-eqEnergyTemp}
\begin{split}
\partial_{t}\text{P}_e=&
-\nabla\cdot\left(\text{P}_e\mathbf U_\text{E}\right)
+2/3\left(\gamma\sigma\text{P}_e\sqrt{\text{T}_e}e^{\Lambda-\Delta\Phi/\text{T}_e}\right.\\
&\left.-\text{P}_e\nabla\cdot\mathbf U_\text{E}
+\chi\nabla^2\text{P}_e
+\mathcal{S}_{\text{T}_e}\right)
\end{split}
\end{align}
\end{refsection}

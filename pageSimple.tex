%%%%%%%%%%%%%%%%%%%%%%%%%%%%%%%%%%%%%%%%%%%%%%%%%%%%%%%%%%
%%
%%  PROJECT: Thèse RFU
%%
%%  Created by Romain Futtersack on 20/01/11.
%%  Copyright 2011 __UPS-CEA__. 
%%  
%%  All rights reserved.
%%
%%%%%%%%%%%%%%%%%%%%%%%%%%%%%%%%%%%%%%%%%%%%%%%%%%%%%%%%%%

\documentclass[12pt]{article}

%%%%%%%%%%%%%%%%%%%%%%%%%%%%%%%%%%%%%%%%%%%%%%%%%%%%%%%%%%
%% 
%% MARK: Packages declarations
%%
%%%%%%%%%%%%%%%%%%%%%%%%%%%%%%%%%%%%%%%%%%%%%%%%%%%%%%%%%%



%%%%%%%%%%%%%%%%%%%%%%%%%%%%%%%%%%%%%%%%%%%%%%%%%%%%%%%%%%
%% 
%% MARK: Fonts declarations
%%
%%%%%%%%%%%%%%%%%%%%%%%%%%%%%%%%%%%%%%%%%%%%%%%%%%%%%%%%%%

%\usepackage[utf8]{inputenc}
\usepackage[T1]{fontenc}
\usepackage{titling}
\setlength{\droptitle}{-145pt}
\usepackage{mathptmx}
\usepackage[english]{babel}
\usepackage{amsmath,amsfonts,amsthm,amssymb}

%%%%%%%%%%%%%%%%%%%%%%%%%%%%%%%%%%%%%%%%%%%%%%%%%%%%%%%%%%
%% 
%% MARK: Input macro file
%% 
%%%%%%%%%%%%%%%%%%%%%%%%%%%%%%%%%%%%%%%%%%%%%%%%%%%%%%%%%%



%%%%%%%%%%%%%%%%%%%%%%%%%%%%%%%%%%%%%%%%%%%%%%%%%%%%%%%%%%
%% 
%% Begin Document
%% 
%%%%%%%%%%%%%%%%%%%%%%%%%%%%%%%%%%%%%%%%%%%%%%%%%%%%%%%%%%

\begin{document}

	%%%%%%%%%%%%%%%%%%%%%%%%%%%%%%%%%%%%%%%%%%%%%%%%%%%%%%%%%%
	%% 
	%% MARK: Title & Data
	%% 
	%%%%%%%%%%%%%%%%%%%%%%%%%%%%%%%%%%%%%%%%%%%%%%%%%%%%%%%%%%
\newcommand{\vecMath}[1]{\mathbf{#1}}

\title{Influence of boundary conditions on 2D interchange turbulence simulations}
\author{\small{R. Futtersack\textsuperscript{1,2},
		P. Tamain\textsuperscript{1}, 
		G. Hagelaar\textsuperscript{2}, 
		Ph. Ghendrih\textsuperscript{1},}\\ 
		\small{J.P. Boeuf\textsuperscript{2}, 
		A. Simonin\textsuperscript{1}}\\
		\scriptsize{\textit{\textsuperscript{1}CEA, IRFM, F-13108 St. Paul-lez-Durance, France}}\\
		\scriptsize{\textit{\textsuperscript{2}Universite Paul Sabatier Toulouse - LAPLACE - 118 route de Narbonne, F-31062 Toulouse cedex 9}}}
\date{}
\maketitle

\thispagestyle{empty}

\small{The experimental fusion reactor ITER will be heated by injection of a fast neutral
beam generated by acceleration and neutralization of negative ions. For this purpose,
new kinds of negative ion sources are currently developed with magnetic fields in the
expansion chamber used either to confine the plasma or to reduce the electron temperature 
near the extraction grid, which favors the negative ion production and survival.
The understanding of magnetized transport is then a major issue in order to build
a consistent model of these ion sources. This goal can be achieved by taking advan-
tage of the great efforts which have already been invested toward the comprehension
of transport phenomena in the Scrape-of-Layer (SOL) of tokamaks. Indeed in these
two cases, open field lines and sheath physics play a key role in the predominantly
turbulent cross-field transport.
\\

In this perspective, the code TOKAM2D has been modified to include flexible
boundary conditions, which is the first step towards the modeling of the ITER ion
source geometry. TOKAM2D is a 2D fluid model for the interchange instability in the
SOL of tokamaks ; the perpendicular transport is described in term of drifts (electric,
diamagnetic and polarization), turbulence is driven by an incoming particle flux and
the parallel boundary conditions, derived from the Bohm criterion, appear through
source terms after integration of the equations in the parallel direction on the basis
of the flute hypothesis. In its original version, the numerical scheme was based on a
second order time-splitting and a semi-spectral resolution for the spatial discretization
: linear terms and derivatives being computed in the Fourier space.
A modified version of the code has been developed, based on a finite volume method
using the MUSCL scheme for the advection terms in order to relax the periodic boundary 
conditions. The implicit advancement of some of the linear terms is still performed
along periodic direction in Fourier space when possible.
\\

Within this presentation, we detail the new numerical scheme implemented in TO\-KAM2D
and the analytical validation of the new version of the code. We focus in particular on
instabilities growth rates. We then presents a comparison of simulations run with periodic
and non-periodic boundary conditions in order to check the influence of boundary
conditions on the turbulent transport. We will also discuss on the future improvements
of the code like the inclusion of energy balances, implicit computation of chosen terms
and implementation of parallel direction.}

\newpage

$$\partial_t{\vecMath{v}_i}+(\vecMath{v}_i\cdot\vecMath{\nabla})\vecMath{v}_i+\nu_i\vecMath{v}_i=\frac{e}{m_i}(-\vecMath{\nabla\phi}+\vecMath{v}_i\times\vecMath{B})-\frac{\vecMath{\nabla P_i}}{m_in}$$
$${\vecMath{v}_i}^{k+1}(\frac{1}{\Delta t}+\nu_i)=\frac{{\vecMath{v}_i}^{k}}{\Delta t}-({\vecMath{v}_i}^{k}\cdot\vecMath{\nabla}){\vecMath{v}_i}^{k}-\frac{e}{m_i}\vecMath{\nabla\phi}-\frac{\vecMath{\nabla P_i}}{m_in}+\frac{eB}{m_i}{\vecMath{v}_i}^{k+1}\times\vecMath{b}$$
$${\vecMath{v}_i}^{k+1}=\frac{\vecMath{W}}{\nu_i^*}+\frac{\omega_{ci}}{\nu_i^*}{\vecMath{v}_i}^{k+1}\times\vecMath{b}$$

$$h_i{\vecMath{v}_i}^{k+1}\times\vecMath{b}=\frac{h_i}{\nu_i^*}\vecMath{W}\times\vecMath{b}-h_i^2\text{.} {\vecMath{v}_i}^{k+1}+h_i^2\text{~}\vecMath{b}\cdot{\vecMath{v}_i}^{k+1}\cdot\vecMath{b}$$
$${\vecMath{v}_i}^{k+1}(1+h_i^2)=\frac{\vecMath{W}}{\nu_i^*}+\frac{h_i}{\nu_i^*}\vecMath{W}\times\vecMath{b}+h_i^2\text{~}\vecMath{b}\cdot{\vecMath{v}_i}^{k+1}\cdot\vecMath{b}$$

$$h_i^2\text{~}\vecMath{b}\cdot{\vecMath{v}_i}^{k+1}\cdot\vecMath{b}=\frac{h_i^2}{\nu_i^*}\vecMath{b}\cdot\vecMath{W}\cdot\vecMath{b}$$
$${\vecMath{v}_i}^{k+1}=\frac{1}{\nu_i^*(1+h_i^2)}(\vecMath{W}+h_i\text{~}\vecMath{W}\times\vecMath{b}+h_i^2\text{~}\vecMath{b}\cdot\vecMath{W}\cdot\vecMath{b})$$



$$v^s_{pol}=-\frac{m_s}{q_sB^2}\left(\frac{d\nabla_\bot U}{dt}\right)_\bot$$

$$\vec{v_{e}}=\frac{\vec{B}\times\vec{\nabla}\Phi}{B^2}$$


$$\frac{m_iu^2_{ieg}}{T_e}\ge 1+\frac{T_i}{T_e} \simeq_{T_i\to0} M_{ieg}\ge 1$$
$$\alpha\partial_2X+\beta\partial X+\gamma X = \delta$$
$$\rho_s^i/L$$

$$D_{Turb}=\frac{\left<\Gamma_x\right>}{\left<\nabla N\right>}$$
$$V^{eff}_N=\frac{\left<\Gamma_x\right>}{\left<N\right>}$$

$$\left<\hat{\phi}(k)\right>_{t}$$
$$\left<\hat{\phi}(\omega)\right>_{y}$$
$$\left<\Gamma_x\right>_{t,y}$$
\newpage
\section{Physique de la turbulence de bord}
\subsection{Moments cinetiques}
Equation cinetique de Vlasov 
$$\partial_t f_s
+\vec{v}_s\cdot\nabla_{\vec{r}}f_s+\frac{\vec{F}_s}{m_s}\cdot\nabla_{\vec{v}}f_s=\partial_tf{|_{coll}}$$
Comme $\nabla_{\vec{v}}\cdot\vec{F}_s=0$ on peut l'ecrire sous sa forme
conservative :
$$\partial_t f_s
+\nabla_{\vec{r}}\left(f_s\vec{v}_s\right)+\nabla_{\vec{v}}\left(\frac{\vec{F}_s}{m_s}f_s\right)=C$$
\subsubsection{Equation de conservation de la matiere}
$$\partial_t\int_v f_s d^3v
+\int_v\vec{v}_s\cdot\nabla_{\vec{r}}f_sd^3v+\int_v\frac{\vec{F}_s}{m_s}\cdot\nabla_{\vec{v}}f_sd^3v=\int_vCd^3v$$

$$\partial_tn+\nabla\left(n\vec{u}\right)=0$$
$$nm\left(\partial_t\vec{u}+\vec{u}\cdot\nabla\vec{u}\right)=nq\left(\vec{E}+\vec{u}\times\vec{B}\right)-\nabla
P$$
$$nm\left(\partial_t\vec{u}_\perp+\vec{u}_\perp\cdot\nabla\vec{u}\right)=nq\left(\vec{E}_\perp+\vec{u}_\perp\times\vec{B}\right)-\nabla_\perp
P$$
$$\vec{u}_\perp\times\vec{B}=\frac{1}{nq}\nabla_\perp P-\vec{E}_\perp$$

$$\vec{u}_E=\frac{\vec{B}\times\nabla_\perp U}{B^2}$$
$$\vec{u}_{dia}=\frac{\vec{B}\times\nabla_\perp P}{nqB^2}$$
$$\vec{u}_{pol}=-\frac{m}{qB^2}\left(\frac{d\nabla_\perp U}{dt}\right)_\perp$$

$$\partial_tn+\nabla_\perp\left(n\vec{u}_\perp\right)+\nabla_\parallel\left(n\vec{u}_\parallel\right)=0$$
$$n\nabla_\perp\vec{u}_E=nB\left[\frac{1}{B^2},U\right]$$
$$\vec{u}_E\nabla_\perp n=\frac{\vec{B}\times\nabla_\perp
U}{B^2}\nabla_\perp\left(n\right)=\vec{B}\frac{\nabla_\perp
U\times\nabla_\perp n}{B^2}=\frac{1}{B}\left[U,n\right]$$
$$\nabla_\perp\left(n\vec{u}_{dia}\right)=\nabla_\perp\frac{\vec{B}\times\nabla_\perp
P}{B^2}=\frac{1}{q}B\left[P,\frac{1}{B^2}\right]$$
$$\nabla_\parallel\left(n\vec{u}_\parallel\right)=B\nabla_\parallel\frac{j_\parallel}{B}=\sigma
N \exp\left(\Lambda-\Phi\right)$$
$$\nabla_\perp B \ll \nabla_\perp n $$
$$E=-\nabla\Phi=E_a+E_d$$
$$nE_a=\frac{\mu_i\nabla p_i-\mu_e\nabla p_e}{\mu_e+\mu_i}-\frac{\mu_e\mu_i}{\mu_e+\mu_i}\vec{B}\times\nabla p$$
$$h_{e,i}=\mu_{e,i}B$$
$$\nabla\cdot\left(\frac{en(\mu_e+\mu_i)}{1+h_eh_i}(\nabla\Phi^{k+1}+E_a)+\frac{h_e-h_i}{1+h_eh_i}b\times j^k+\frac{h_eh_i}{1+h_eh_i}j^k_{//}\right)$$



\end{document}
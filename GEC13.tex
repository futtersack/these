% %This is a very basic article template.
% %There is just one section and two subsections.
\documentclass[a4paper,12pt]{article} % classe "article", papier a4 et police de
% 11pt

\usepackage[utf8]{inputenc} \usepackage[T1]{fontenc}
\usepackage{cite}
\usepackage{mathptmx}
\usepackage[english]{babel}
\usepackage{amsmath,amsfonts,amsthm,amssymb}
\usepackage{graphicx}
\usepackage{multicol}
\usepackage{abstract}
\usepackage[center,labelfont=bf]{caption}
\usepackage{tocloft}
\usepackage{indentfirst}

\usepackage[margin=25mm]{geometry}

\begin{document}



\begin{abstract}

While various low-pressure plasma sources operating with a steady magnetic field are widely used in industrial and research 
applications, the knowledge of magnetized transport in these non-thermal plasmas is still incomplete. As the transport of 
charges and currents in such plasma sources may show a complex and ill-understood
behavior, we investigate the issue of magnetized transport as such. A new 2D dynamical fluid 
model has been developed, combining the usual methods of cold plasma modeling with the drift velocity approximation 
technique drawn from hot fusion plasmas research, and therefore allowing to explore a large range of magnetic field 
strengths and topologies. 
We then analyze simulations related to representative experiments with various magnetic field configurations in order 
to characterize the transport in these low-temperature plasmas and compare the results with experimental data and 
application-oriented models. 
For two different negative ion sources, the main behavior of the plasma is recovered, with the emergence of asymmetries 
due to the drifts induced by the magnetic field. The model is also able to capture the transient dynamics of the plasma 
such as certain types of instabilities. 

\end{abstract}


\end{document}

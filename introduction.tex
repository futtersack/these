\Chapter{Introduction}
\label{Introduction}
\begin{refsection}

\section{Fondamentaux et échelles caractéristiques}
\emph{Le plasma}. Le plasma est souvent considéré un peu équivoquement comme le
quatrième état de la matière. Dans notre environnement proche, nous avons pris
conscience de son existence à travers les phénomènes de flammes, d'éclairs,
d'aurore boréales ou d'arc électriques. Mais à des conditions de pressions et de
températures différentes de celles de notre atmosphère terrestre, il est
omniprésent : plus de 99\% de la matière connue est sous cette forme.

Une définition plus adaptée du plasma est celle d'un gaz conducteur. Une partie
des atomes le composant est ionisée, donnant naissance à une population
d'électrons libres et d'ions de différentes espèces. Ces populations permettent
alors le transport de courant et, sensibles aux forces électromagnétiques,
influencent fortement le comportement global du plasma en provoquant des
phénomènes collectifs, non-linéaires et turbulents.

Le plasma et son comportement sont décrits par la théorie de la physique des
plasmas. Elle intègre les connaissances de nombreux domaines, tels que la
physique statitique, l'électromagnétisme, ou encore la dynamique des fluides.

\subsection{Les paramètres plasmas}
Les plasmas se présentent donc comme des gaz dont le comportement est influencé
de manière non négligeable par une population d'électrons libres. Cette population
se caractérise par deux paramètres principaux :

\begin{itemize}
  \item sa densité, $n_e$, mesurant le nombre de particules par élément de volume
  \item sa température électronique $T_e$, mesurée en électronvolts, représentative
   de l'energie cinétique moyenne non dirigée des électrons (aussi appelée
   agitation thermique)
\end{itemize}

\begin{equation}
	eT_\text{e}=m_\text{e} v_{\text{T}_\text{e}}\puissance{2}
\end{equation}

où $m_\text{e}$ est la masse de l'électron, et $e$ la charge élémentaire. Pour un électron possédant
une énergie de \unit{1}{\electronvolt}, la vitesse thermique est de l'ordre de
\unit{400}{\kilo\meter}/s.
Du fait de la différence de masse, cette vitesse est bien inférieure pour les
ions, qui à même énergie auront une vitesse thermique d'environ
\unit{10}{\kilo\meter}/s.

\begin{equation}
	v_{\text{T}_\text{i}}\sim\left(m_\text{e}/m_\text{i}\right)^{\text{\textonehalf}}
	v_{\text{T}_\text{e}}
\end{equation}

La figure (\ref{zoologie}) représente une classification des plasmas
en fonction de ces deux paramètres principaux qui vont influer sur la dynamique du
transport des particules et du courant.
La théorie présentée dans la suite de cette thèse ne concerne que les plasmas
dits classiques :

\begin{itemize}
  \item les plasmas naturels peu dense tels que l'espace interstellaire,
  le vent solaire, la magnétosphère, et l'ionosphère
  \item les plasmas naturels denses tels que les éclairs et les étoiles
  \item les plasmas industriels, de laboratoire, et thermonucléaires
\end{itemize}

\begin{figure}[htbp]
\centering
\includegraphics[height=80mm,width=64mm]{figures/zoologie.png}{\caption{Classification
de différents plasmas en fonction de $n_e$ et $T_e$ issue du livre du National
Research Council \parencite{NRC}}\label{zoologie}}
\end{figure}

Dans ces plasmas, le dégré d'ionisation $\alpha$ est donné par le rapport
entre la densité électronique $n_\text{e}$ et la densité de gaz neutre
$n_\text{n}$ :

\begin{equation}
\alpha=\frac{n_\text{e}}{n_\text{e}+n_\text{n}}
\end{equation}

Cette fraction va définir l'importance de l'interaction entre les particules
neutres et les particules chargées. Cependant, même à très faible $\alpha$,
l'apparition d'une population de porteurs de charge va modifier considérablement 
les caractéristiques et la dynamique du plasma.

L'ionisation du gaz suit l'évolution de la température électronique $T_\text{e}$
qui mesure l'agitation thermique des électrons. Densité et température
électroniques permettent de définir le paramètre plasma:

\begin{equation}
\label{1-paramPlasma}
\Xi=\frac{<E_\text{p}>}{<E_\text{c}>}=\frac{e\puissance{2}
n_\text{e}\puissance{1/3}}{\varepsilon\indice{0} eT_\text{e}}
\end{equation}

Le paramètre plasma représente le ratio entre l'énergie thermique des
électrons et leur énergie potentielle électrostatique coulombienne, ie.
l'agitation thermique desordonnée contre les forces d'interactions
coulombiennes structurantes.

Les plasmas classiques (ou cinétiques) sont caractérisés par
$\Xi\ll 1$. Ils ont une population d'électrons assez espacée et/ou une
température suffisamment élevée.

\subsection{Quasineutralité et phénomène d'écrantage}
La dynamique d'un plasma résulte du couplage entre le mouvement des
particules chargées et les forces électromagnétiques présentent ou qui se
forment dans le système :

\begin{equation}
\frac{\text{d} \mathbf
v_\alpha}{\text{dt}}=\frac{q_\alpha}{m_\alpha}(\mathbf E+ \mathbf
v_\alpha\times\mathbf B)
\end{equation}

l'indice $\alpha$ dénotant le type de la particule chargée, $m_\alpha$
et $q_\alpha$ sa masse et sa charge, $\mathbf v_\alpha$ sa vitesse et $(\mathbf
E,\mathbf B)$ le champ électromagnétique.
Elle se décompose principalement sur deux échelles de temps :
la première phase correspond à l'équilibre électrostatique, où une
quasineutralité s'établit au sein du plasma. Dans une deuxième phase, l'évolution quasineutre du
plasma, plus lente, tend à former un équilibre entre le déplacement des particules et la valeur locale
et instantanée des champs de force extérieurs.

L'un des processus les plus rapides que l'on peut trouver dans un plasma est lié à
l'équilibre microscopique électrostatique qui s'opère
entre l'agitation thermique et l'interaction coulombienne des particules. Les
ions et les électrons issus de l'ionisation se réorganisent spontanément pour former un ensemble électriquement neutre.
Cette dynamique est essentiellement portée par les électrons et décrite aux plus petites
échelles spatiotemporelles par le principe fondamental de la dynamique et la loi
de Boltzmann-Poisson :

\begin{align}
\label{1-PFD}
m_\text{e}\frac{\text{d} \mathbf{v}_\text{e}}{\text{d} t}=-e\mathbf E
\;\;\;\;\;\;\text{et}\;\;\;\;\;\;
\nabla\cdot\mathbf{E}=\rho\varepsilon\indice{0}\puissance{-1}
\end{align}

où $\mathbf{v}_\text{e}$ est la vitesse de l'électron et
$\rho=e\,(n_\text{i}-n_\text{e})$ la densité spatiale de charge.
Tout écart entre la densité d'ions et la densité d'électrons
entraîne ainsi l'apparition d'un fort champ électrique de rappel à l'équilibre.

Les échelles fondamentales définies par \eqref{1-PFD} sont la pulsation
plasma et la longueur de Debye, reliée par l'intermédiaire la vitesse thermique :

\begin{equation}
\omega_\text{p}\left[\left(\frac{n\indice{0}
e\puissance{2}}{\varepsilon\indice{0} m_\text{e}}\right)^{\text{\textonehalf}}
\right]\cdot\lambda_D\left[\left(\frac{\varepsilon\indice{0}
eT_\text{e}}{n\indice{0} e\puissance{2}}\right)^{\text{\textonehalf}}\right]
=v_{\text{T}_\text{e}}\left[\left(\frac{eT_\text{e}^{^{\phantom{1}}}}{m_\text{e}}\right)^{\text{\textonehalf}}\right]
\end{equation}

La pulsation plasma $\omega_\text{p}$ correspond à la fréquence typique
d'oscillation des électrons en réponse à une petite séparation de charge. Des processus plus lents que cette fréquence ne voient
pas apparaître d'équarts à la neutralité. 

La longueur de Debye $\lambda_\text{D}$, quant à elle, permet de définir le
phénomène d'écrantage :
C'est une longueur au-delà de laquelle le champ électrique créé par une particule est \emph{écranté} par 
un nuage de particules de charge opposée. Dans un plasma chaque particule est écrantée et participe
à l'écrantage des autres ; Le nombre de particules présentes
dans la sphère de rayon $\lambda_\text{D}$ définit alors un paramètre
adimensionné\footnote{Un
grand nombre de particules dans la sphère de Debye $4\pi/3\,n\lambda_D^3\gg1$
définit un plasma idéal, qui peut alors se décrire comme un plasma où le
phénomène d'écrantage électrique est dominant.} semblable au paramètre plasma $\Xi$ \eqrefp{1-paramPlasma}.

Au delà de ces échelles, le plasma est dans un état de \emph{quasi-neutralité}
\footnote{La quasi-neutralité n'est pas la seule conséquence de ce phénomène de réorganisation
de charges : lors de l'application d'un champ électrique ou magnétique, celui-ci est rapidement
écranté par le plasma, les particules se
déplaçant afin de créer un champ opposé.}, 
ie. où :

\begin{equation}
n_\text{e}=n_\text{i}
\label{quasineutralité}
\end{equation}

L'interaction coulombienne d'une particule chargée avec l'ensemble des autres particules peut alors
se réduire à son intéraction avec un potentiel électrique local $\Phi$,
résultant de la somme des potentiels microscopiques individuels.

Le système évolue ensuite sur cet équilibre par le biais de
\emph{phénomènes de transport} résultants de l'influence
globale des champs électromagnétiques et de la tendance à
l'homogénéisation du plasma par collisions entres particules.
Ces processus interviennent sur des échelles relativements
lentes et relativement grandes devant $\omega_\text{p}$ et $\lambda_\text{D}$ :

\begin{equation}
\text{T}\gg \omega_\text{p}\puissance{-1}
\;\;\;\;\text{et}\;\;\;\;\text{TV}\gg\lambda_\text{D}
\end{equation}

$\text{T}$ étant le temps et $\text{V} $ la vitesse, caractériques du processus
de transport considéré.
On parle alors de diffusion/convection de particules, de viscosité (transfert de
quantité de mouvement), de conductivité (transport de charges) ou encore de
conductivité thermique (transport de chaleur).

\subsection{Collisions}
En plus de l'interaction coulombienne globale de l'ensemble des espèces chargées, 
les particules, prises deux à deux, peuvent interagir binairement. 
Ces interactions directes entre particules donnent lieu à des transferts
d'impulsion, d'énergie ou encore à des réactions plus complexes telles que
l'ionisation, l'exitation et la recombinaison ; elles sont regroupées
sous le terme général de collisions et se caractérisent par des fréquences de
collisions.

Lors de collisions \emph{élastiques}\footnote{Les collisions sont dites élastiques car
elles donnent lieu à des transferts d'impulsion et d'énergie sans changer
la nature des particules à l'issue de la collision.}, les particules échangent de l'énergie 
en fonction de leur ratio de masse. Entre les électrons et les espèces plus
lourdes, ce transfert d'énergie est très faible et cause une isotropisation
globale des vitesses électroniques.
Dans le cas des plasmas totalement ionisés, seules les collisions coulombiennes 
(interactions binaires entre particules chargées dues à la force de Coulomb)
sont présentes. 
La féquence de collision $\nu_\text{ei}$ mesure alors la fréquence à laquelle un
électron est dévié de plus de \unit{90}{\degree} de sa trajectoire initiale:

\begin{equation}
	\nu_\text{ei}\sim\frac{4\sqrt{2\pi}e\puissance{4} n_\text{e} \ln\Lambda
	}{3(4\pi\varepsilon\indice{0})\puissance{2}m_\text{e}\puissance{1/2}T_\text{e}\puissance{3/2}}
\end{equation}

avec $\ln \Lambda$ le logarithme de Coulomb,
$\Lambda=\lambda_\text{D}(e\puissance{2}/4\pi\varepsilon\indice{0}
T_\text{e})\puissance{-1}$ étant le ratio entre la longueur de Debye et une
distance minimale d'approche.

Dans les plasmas faiblement ionisés, le
gaz neutre forme un fond diffus et influence fortement le transport des particules chargées.
Les fréquences de collisions électrons-neutres et ions-neutres sont alors
définies par :

\begin{equation}
	\nu_\text{en}=\sigma_\text{en} n_\text{n} v_\text{e}
	\;\;\;\;\text{et}\;\;\;\;\nu_\text{in}=\sigma_\text{in} n_\text{n} v_\text{i}
\end{equation} 
 
avec $\sigma_\text{en}$ et $\sigma_\text{in}$ les sections efficaces de
réaction\footnote{\fcolorbox{red}{white}{Les sections efficaces de collisions sont permettent de
caractériser \ldots}} électrons-neutres et ions neutres. La principale
différence dans le transport entre les plasmas faiblement et totalement ionisés tient à la
présence de ce gaz neutre : les collisions avec le gaz sont un puit
direct de quantité de mouvement et d'énergie alors que les
collisions coulombiennes résultent en une redistribution de celles-ci au sein
du plasma.

Les collisions avec le gaz peuvent aussi entraîner des réactions plus complexes
au cours desquelles les particules changent de nature ou d'état (ionisation,
attachement électronique, excitation\ldots).
Ces collisions sont dites \emph{inélastiques} en opposition aux collisions
élastiques. Parmis ces réactions, l'ionisation est le processus le plus
important car il contraint le bilan de particule et contrôle donc l'existence
du plasma. L'association de fréquences de collisions à ces types de réaction
(eg. $\nu_\text{iz}$ pour l'ionisation), là aussi en fonction de sections
efficaces, permet alors un ordering sur l'importance des processus dans l'évolution du plasma.

En plus d'une fréquence, on associe souvent une longueur caractéristique
à l'occurence d'une collision. Le libre parcours moyen $\lambda_\text{lpm}$
représente alors la distance que peut parcourir une particule à une vitesse
$v$ avant de subir une certaine interaction avec le système qui l'entoure :

\begin{equation}
	\lambda_\text{lpm}=\frac{v}{\nu}
\end{equation} 

La collisionnalité d'une espèce est
définie par le ratio entre son libre parcours moyen $\lambda_\text{lpm}$ et la
taille $L$ du plasma. Quand le libre parcours moyen est petit devant la
taille du plasma $\lambda_\text{lpm}/L\ll 1$
les particules ont le temps de faire de nombreuses collisions avant de
traverser entièrement le système. Dans des plasmas collisionnels, la description
du transport des espèces est alors simplifiée, les collisions amenant le système dans un état
d'équilibre statistique, caractérisé par des fonctions de distributions de
type Maxwell-Boltzmann (cf. paragraphe \ref{Introduction}-\ref{Maxwell-Boltzmann}).

\fcolorbox{red}{white}{encore beaucoup de choses à
dire\ldots}

\subsection{Plasma et champ magnétique} 
Un plasma magnétisé est un plasma dans lequel le champ magnétique ambiant
$\mathbf{B}$ est suffisement important pour modifier de manière non-négligeable
le transport des particules. Les plasmas magnétisés sont fortement
anisotropiques, réagissant différement aux forces parallèlement et
perpendiculairement au champ magnétique.

Dans un champ magnétique uniforme, le mouvement d'une particule chargée
soumise à la force de Lorentz peut se décrire par deux composantes :

\begin{itemize}
  \item Un déplacement libre le long lignes de champ
  \item Un mouvement de rotation rapide de la particule
  autour d'un centre-guide dans le plan perpendiculaire à l'axe magnétique
\end{itemize}

Ce mouvement de rotation autour des lignes de champ, appelé mouvement
cyclotronique, est caractérisé par la pulsation cyclotronique : 

\begin{equation}
\omega\indice{c_{\alpha}}=\frac{e B}{m_\alpha}
\end{equation}

Le rayon de Larmor $\rho_\text{L}$ de l'orbite révèle quant à lui une échelle
spatiale représentative du transport magnétisé. Le rayon de Larmor se
calcule en fonction de la pulsation cyclotronique et de la vitesse
$v\indice{\alpha}$ de la particule :

\begin{equation}
\rho\indice{L_{\alpha}}
=\frac{v\indice{\alpha}}{\omega_{c_{\alpha}}}
\end{equation}

Pour les ions suffisement énergétiques, ainsi que pour les électrons en général,
la vitesse typique de la particule est sa vitesse thermique. On peut alors noter :

\begin{equation}
\omega\indice{c_{\alpha}}\left[\frac{e B}{m_\alpha}\right]
\cdot\rho\indice{L_{\alpha}}\left[\frac{\sqrt{m_\alpha eT_\alpha}}{eB}\right]
=v\indice{T_{\alpha}}\left[\left(\frac{eT_{\alpha}}{m_{\alpha}}\right)^{\text{\textonehalf}}\right]
\end{equation}

Quand le champ magnétique s'intensifie, les trajectoires
hélicoïdales se resserrent, limitant de plus en plus le déplacement des
particules perpendiculairement à la direction magnétique. L'influence du champ
sur le transport peut alors être mesuré par le paramètre de
magnétisation $\delta$, une espèce pouvant être considérée comme magnétisée à
partir du moment où la longueur caractéristique $L$ du transport considéré est
grande devant son rayon de Larmor :

\begin{equation}
\delta=\frac{\rho_{L_{\alpha}}}{L}\ll 1
\end{equation}

\begin{figure}[htbp]
\centering
\includegraphics[width=0.8\textwidth]{figures/mouvementCyclotron.png}
{\caption{Mouvement cyclotronique des particules autour des lignes de champ
magnétique.}\label{1-particleDrifts}}
\end{figure}

L'application d'un champ électrique conjointement à un champ magnétique
induit une troisième sorte de mouvement, appelé communément "dérive
ExB" (E-cross-B) : le phénomène de dérive se décrit comme un lent déplacement du
centre guide de la particule, perpendiculairement aux champs électrique et
magnétique (cf. figure \ref{1-particleDrifts}). Son expression s'obtient en
prenant le produit vectoriel de l'équation du mouvement par le champ magnétique
:

\begin{equation}
\mathbf{v}_\text{E}=\frac{\mathbf{E}\times\mathbf{B}}{B^2}
\end{equation}

La vitesse de dérive électrique, principale source de transport perpendiculaire
dans un plasma magnétisé, est indépendante de la masse et de la charge des
particules : égale pour les ions et les électrons, elle n'est donc à
l'origine d'aucun courant, ce qui la rend particulière. 

Il peut en effet apparaître d'autres vitesses de dérive. En toute généralité, 
pour chaque force appliquée au plasma, le champ magnétique est à l'origine
d'une vitesse de dérive perpendiculaires aux deux forces en jeu :

\begin{equation}
\mathbf{v}\indice{\text{F}}=\frac{\mathbf{F}\times\mathbf{B}}{q_\alpha B^2}
\end{equation}

Si la force $\mathbf F$ est indépendante de la charge de la particule ou
fonction de la masse de la particule, la différence entre les vitesses de dérive
électroniques et ioniques induira un courant $\mathbf
j\indice{\text{F}}=en(\mathbf v\indice{\text{F}_\text{i}}-\mathbf
v\indice{\text{F}_\text{e}})$. 

\fcolorbox{red}{white}{
\begin{minipage}{0.9\textwidth}Dans un plasma, où l'anisotropie du champ
magnétique est souvent utilisée pour contrôler et limiter le déplacement des
particules, les vitesses de dérives sont souvent à l'origine d'un déconfinement
entraîné par le développement de comportements
non-linéaires complexes\ldots
\end{minipage}}
\subsection{Les gaines électrostatiques}
Un plasma est souvent, à un moment ou à un autre, amené à entrer en contact avec
une paroi ou un quelconque obstable matériel. L'interaction entre
les deux milieux conditionne alors l'existence du plasma et influence
fortement son comportement. Une paroi agit en effet comme un parfait
puit de matière et d'énergie pour le plasma, les ions et les électrons se
recombinant facilement à sa surface. Ceci étant, un plasma ne peut se maintenir
que si le nombre de particules créées dans son volume est suffisant pour
balancer ces pertes en surface. 

De plus, les électrons interceptant généralement
plus rapidement la paroi du fait de leur faible masse et de leur température,
polarisent négativement la surface du matériau.
L'augmentation progressive de ce potentiel mène alors à la formation en
périphérie du plasma d'une région de densité ionique supérieure, et donc
non-neutre, que l'on appelle gaine électrostatique (ou gaine de Debye, voir
figure~\ref{1-gaine}).

\begin{figure}[htbp]
\centering
\includegraphics[height=60mm,width=60mm]{figures/sheath1.jpg}{\caption{Séparation
du plasma entre une région quasineutre et la gaine à sa
périphérie. A la frontière, il y a rupture de la quasineutralité et la vitesse
ionique est égale à la vitesse de Bohm
$C_\text{s}$\parencite{Rax}.}\label{1-gaine}}
\end{figure}

Dans les plasmas classiques où la taille de la gaine est petite
à l'échelle du plasma, son rôle est essentiellement de
maintenir la quasineutralité dans le volume en régulant les flux d'ions et
d'électrons tombant aux parois. Les électrons, énergétiquement confinés par le
potentiel de la surface, forment un équilibre électrostatique à l'intérieur du
plasma, et peuvent donc être décrits par une distribution de Boltzmann :

\begin{equation}
	n_\text{e}=n\indice{0}\exp(e\Phi/eT_\text{e})
\end{equation}

avec $n\indice{0}$ la densité de référence prise au niveau du potentiel nul.
Pour égaler le flux électronique et écranter la polarisation de la paroi, la
quasineutralité doit être brisée. L'étude de la transition plasma-gaine montre,
à travers l'énoncé du critère de Bohm \parencite{Stangeby}, que
cette condition ne peut être remplie que si la vitesse ionique en entrée de
gaine est au moins égale à $C_s$, communément appelée vitesse de Bohm ou encore
vitesse acoustique ionique\footnote{la deuxième appellation provient de la similitude entre la
transition subsonique-supersonique qui s'observe dans les gaz neutres. Les
ondes acoustiques, reliées aux surpressions et aux souspressions, se propagent
dans un plasma par le champ électrique. L'élasticité du plasma est alors due à
la température électronique $T_\text{e}$ et à l'inertie ionique.} :

\begin{equation}
C_\text{s}=\sqrt{\frac{e (T_\text{e}+T_\text{i})}{m_\text{i}}}
\end{equation}

\subsubsection{Physique de la gaine}

Considérons de nouveau l'équation de Poisson, qui contrôle l'évolution du
potentiel électrique, au voisinage de la paroi :
\begin{equation}
\frac{d\puissance{2}\Phi}{d\xi}=\frac{1}{\lambda_D\puissance{2}}\frac{n_\text{e}-n_\text{i}}{n\indice{0}}
\end{equation}
 où $\Phi$ est le potentiel normalisé $\Phi=eU/T_\text{e}$, $\xi$ est la
 direction normale à la paroi, $n_\text{e}$ et $n_\text{i}$ les densités
 électronique et ionique respectivement, et $n\indice{0}$ la densité
 moyenne du plasma. Quand le plasma est suffisement dense pour que $\lambda_D\ll
 L$, l'équation de poisson permet d'estimer le champ électrique des deux
 régions :

 \begin{itemize}
   \item Face à la paroi chargée négativement se
   développe un fort gradient de potentiel ($d\Phi/d\xi\sim 1/\lambda_D$) dont le plasma
   s'écrante en formant une épaisseur de peau ionique non-neutre de
   taille caractérisique $\lambda_D$
   \item Le reste du plasma, de taille caractéristique $L$, satisfait
   à l'hypothèse de quasineutralité et le potentiel varie en
   $d\Phi/d\xi\sim1/L$
 \end{itemize}

La frontière entre la gaine et la partie
quasineutre du plasma, qualifiée de prégaine, peut être définie à partir de la
brisure de la quasineutralité. A l'intérieur de la gaine, afin de maintenir la
décroissance du potentiel jusqu'à la paroi, la densité ionique doit diminuer
moins rapidement que la densité d'électrons. 

\begin{equation}
	\frac{dn_\text{i}}{d\xi}<\frac{dn_\text{e}}{d\xi}
\end{equation}

En applicant la loi de conservation de l'énergie entre
l'entrée de la gaine (où les densités sont égales) et la paroi, on trouve le
critère général obtenu par Bohm en hydrodynamique :

\begin{equation}
	v_{\text{i}\,|_{\xi=\xi_s}}\geq C_\text{s}
\end{equation}

où $\xi_s$ indique la position de la frontière plasma/gaine.
Dans le cas de plasmas non-magnétisés et avec l'hypothèse d'une température
électronique uniforme $T_\text{e}\gg T_\text{i}$, on montre de plus que la
vitesse ionique doit rester subsonique dans la prégaine, donnant une limite dure
$v_\text{i}=C_\text{s}$.
Pour estimer la chute de potentiel, on peut alors écrire :

\begin{equation}
	\left(\frac{2e(\Phi_{\xi}-\Phi_{\xi_\text{s}})}{m_\text{i}}+v_\text{i}\puissance{2}\right)_{\xi=\xi_s}=
	\frac{eT_\text{e}}{m_\text{i}}=\left(\frac{2e(\Phi_{\xi}-\Phi_{\xi_\text{s}})}{m_\text{i}}\right)_{\xi=0}
\end{equation}

A partir d'une région centrale où $v_\text{i}=0$, le potentiel chute donc de
$T_{e}/2$ et la densité, par conséquence, d'un facteur
$n_\text{s}/n\indice{0}=e^{-1/2}$, l'indice $s$ dénotant dans la suite du
paragraphe la position de la gaine.
Le flux d'électron atteignant la paroi se calcule en intégrant une maxwellienne sur le
demi-espace des vitesses positives :

\begin{equation}
	v_\text{e}^\text{wall}=\frac{1}{\sqrt{\pi}v_\perp}\int_0^\infty
	v_\perp
	e^{-v_\perp^2/v_\text{T}^2}dv_\perp=\frac{1}{2\sqrt{\pi}}v_\text{T}=\sqrt{\frac{eT_\text{e}}{2\pi
	m_\text{e}}}
\end{equation}

les flux ionique et électronique sont égaux à la paroi, le flux ionique étant
conservé à travers la gaine :

\begin{equation}
	\Gamma_\text{e}^\text{wall}=n_\text{e}^\text{wall}\exp{\left(\frac{\Phi_\text{wall}}{T_\text{e}}\right)}\sqrt{\frac{eT_\text{e}}{2\pi
	m_\text{e}}}=\Gamma_\text{i}^\text{wall}=\Gamma_\text{i}
\end{equation}
\begin{equation}
\label{1-densitéIonique}
	n_\text{i}=n_\text{s}\exp{\left(\frac{\Phi_\text{s}-\Phi_\text{wall}}{T_\text{e}}\right)}\sqrt{\frac{eT_\text{e}}{2\pi
	m_\text{e}v_\text{i}\puissance{2}}}
\end{equation}

en prenant la relation \ref{1-densitéIonique} au seuil de la gaine, on peut
calculer le potentiel flottant $\Lambda$ qui s'établit naturellement entre le
plasma et le mur :

\begin{equation}
	\Lambda=\Phi_\text{s}-\Phi_\text{wall}=\frac{1}{2}T_\text{e}\ln{\frac{m_\text{i}C_\text{s}^2}{2\pi
	m_\text{e}v_\text{i}\puissance{2}}}
	\end{equation}

Si l'on suppose donc $v_\text{i}=C_\text{s}$ à l'entrée de gaine, le potentiel
flottant d'un plasma d'hydrogène à une température électronique d'1eV est
défini par le logarithme du rapport de masse :

\begin{equation}
	\Lambda=\frac{1}{2}\ln{\frac{m_\text{i}}{2\pi
	m_\text{e}}}=2.8388
	\end{equation}
	
\begin{figure}[htbp]
\centering
\includegraphics[height=70mm,width=80mm]{figures/sheath2.jpg}{\caption{Profils
des densités électronique, ionique,
du potentiel et de la vitesse ionique à
la transition plasma-gaine\parencite{Rax}. La taille
de la gaine est ici exagérée.}\label{1-profilesgaine}}
\end{figure}

Pour résumer ces considérations, les profils de densité,
potentiel et vitesse ionique pour un plasma sont tracés sur la figure
(\ref{1-profilesgaine}).

$\Lambda$ est le potentiel correspondant à une égalité stricte des flux au
niveau de la gaine. Cependant, en général il ne définit qu'une position
d'équilibre autour de laquelle le potentiel s'ajuste constament pour
équilibrer les flux à travers la circulation de courant :

\begin{equation}
	j_\text{s}=en_\text{s}v_\text{i}\left(1-e^{\Lambda-\Phi_\text{s}/T_\text{e}}\right)
\end{equation}

\fcolorbox{red}{white}{
\begin{minipage}{0.9\textwidth}flux d'énergie et de chaleur\ldots
\end{minipage}}



\subsubsection{Gaine dans les plasmas magnétisés}
La situation dans le cas d'un champ magnétique normal à la paroi est
identique au cas non-magnétisé. La théorie peut de plus être étendue 
à une incidence $\alpha_\text{inc}$ non-rasante des lignes de champ où une
prégaine magnétique de quelques rayons de Larmor accélère les ions jusqu'à la
vitesse de Bohm via une déflexion imposée à la trajectoire des particules. Le
critère de Bohm est alors légèrement corrigé pour tenir compte du transport tansverse
\cite{Stangeby} :

\begin{equation}
	v_\para\geq
	C_{s}-\frac{v_\perp}{\tan{\alpha_\text{inc}}}
\end{equation}
 
 $v_\perp$ désignant la composante de la vitesse ionique perpendiculaire à la
 surface. 
 
 
 \fcolorbox{red}{white}{
\begin{minipage}{0.9\textwidth}Le cas d'une gaine parallèle au champ magnétique s'avère être bien
 plus complexe et sera abordé dans la suite du manuscrit.
\end{minipage}}

\section{Equations fluides et phénomènes de transport}
\label{Maxwell-Boltzmann}

Pour décrire l'évolution d'un plasma, une possibilité serait de suivre le
déplacement de chacune des particules ainsi que l'ensemble de leurs intéractions
au cours du temp. Ce problème d'interaction à N-corps est cependant bien trop
compliqué, et en fait plutôt superflu quand on considère un si grand nombre de
particule. 

Une autre possibilité est de raisonner statistiquement. La description
microscopique du plasma est alors donnée par un ensemble de fonctions de
distribution $f_\alpha(\mathbf{r},\mathbf{v},t)$, correspondant chacune à une
espèce du plasma. Les coordonnées, six dans l'espace des
phases (trois pour la position et trois pour la vitesse), ainsi que le temps,
sont alors considérées comme sept variables indépendantes.

Pour l'étude de l'évolution du plasma sur les échelles relativement
grandes qui rendent compte des phénomènes de transport, la connaissance
détaillée des fonctions de distribution n'est cependant pas indispensable et le
calcul de moyennes statistiques - les moments en vitesse des fonctions de
distribution - contiennent souvent l'ensemble des informations nécéssaires à la
description du plasma.

\subsection{De la statistique au transport fluide}

\subsubsection{L'équation de Boltzmann}
Le nombre de particules dans un plasma étant généralement élevé, il est
approprié de les représenter d'une manière statistique et en fonction de chaque
espèce, comme un continuum défini par une fonction de
distribution $f$ dénombrant le nombre de particule présentes à un instant $t$
dans un volume de l'espace des phases, ie. $f_\alpha(\mathbf r,\mathbf
v,t)\text{d}\puissance{3}\mathbf r\text{d}\puissance{3}\mathbf v$.
L'évolution de cette fonction de distribution sous l'influence de forces
extérieures et des collisions est alors décrite par l'équation cinétique de
Boltzmann :

\begin{equation}
\label{1-BoltzmannEquation}
\partial_tf_\alpha+\mathbf{v}_\alpha\cdot\nabla_\mathbf{r}f_\alpha+
\frac{\mathbf{F_\alpha}}{m_\alpha}\cdot\vec\nabla_{\mathbf{v}}f_\alpha
=C[f_\alpha]
\end{equation}

où l'indice $\alpha$ dénote les différentes espèces\footnote{La fonction de
distribution peut aussi servir à suivre les différents états quantiques des
ions, des atomes et des molécules (rotationnel, vibrationnel, électronique).},
$\vec\nabla$ est l'opérateur gradient appliqué à l'espace des positions ou des
vitesses, et $\mathbf{F_\alpha}$ se résumant souvent aux forces
électromagnétiques. Lors de collisions, les particules peuvent changer de
trajectectoire instantanément, et ainsi disparaître puis apparaître dans des
régions éloignées de leur espace des phases (ou encore changer d'espèce avec les
collisions inélastiques). L'opérateur de collisions $C[f_\alpha]$ représente
l'intégralité des ces effets collisionnels et se réécrit en
tant que somme sur l'ensemble des espèces :

\begin{equation}
C[f_\alpha]=\sum_\beta C_{\alpha\beta}(f_\alpha,f_\beta)
\end{equation}

$C_{\alpha\beta}$ étant un taux caractéristique correspondant aux collisions
entre chaque espèce, et au sein d'une même espèce. En l'absence collisions,
\eqref{1-BoltzmannEquation} est connue sous le nom de \emph{Equation de Vlasov}.

\subsubsection{Les moments fluides}
L'approche fluide consiste à décrire le comportement des particules de chaque
espèce en terme de quantités macroscopiques telles que la densité $n$, la
vitesse moyenne $\mathbf u$ et la température $T$. Ces quantités sont dérivées
des moyennes des fonctions de distributions sur l'espace des vitesses et leur
évolution est contrôlée par les équations fluides obtenues en intégrant
l'équation de Boltzann multipliée par une certaine puissance de $\mathbf v$.

Le nombre de degré de liberté du problème est ainsi
réduit en ne conservant de l'état du plasma que ses dimensions spatiales et temporelle.
Cette description sous-entend cependant que le plasma est à l'équilibre
thermodynamique local (ETL), ie. que la fonction des vitesses de chaque espèce est
supposée Maxwellienne :

\begin{equation}
f(\mathbf v)=\sqrt{\left(\frac{m_\alpha}{2\pi
eT_\alpha}\right)\puissance{3}}4\pi\mathbf
v\puissance{2}\exp{\left(\frac{-m_\alpha\mathbf
v\puissance{2}}{2eT_\alpha}\right)}
\end{equation}

Le moment en vitesse d'ordre $k$ est un tenseur d'ordre $k$ et s'écrit :
\begin{equation}
M^{(k)}_\alpha(\mathbf
r,t)=\int\mathbf{v}_\alpha^kf_\alpha(\mathbf r,\mathbf v,t)
\,\text{d}\puissance{3}\mathbf{v}
\end{equation}

La densité, définie par le moment d'ordre zéro, est la quantité macroscopique la
plus importante :

\begin{equation}
n_\alpha(\mathbf
r,t)=\int f_\alpha(\mathbf r,\mathbf v,t)
\,\text{d}\puissance{3}\mathbf{v}
\end{equation}
\fcolorbox{red}{white}{
\begin{minipage}{0.9\textwidth}En effet, si l'on ne tient pas compte de la distribution en vitesse, le nombre
de particules dans un volume $\text{d}\puissance{3}\mathbf r$ correspond bien à
la définition d'une densité.\ldots
\end{minipage}}

Les moments d'ordre un (le flux de
particule) et d'ordre deux (la densité d'énergie chaotique), permettent de
définir la vitesse fluide et la température d'une espèce:

\begin{equation}
\Gamma_\alpha(\mathbf
r,t)=n_\alpha\mathbf u_\alpha=\int \mathbf v_\alpha f_\alpha(\mathbf r,\mathbf
v,t) \,\text{d}\puissance{3}\mathbf{v}
\end{equation}

\begin{equation}
\frac{3}{2}n_\alpha T_\alpha(\mathbf
r,t)=\frac{m_\alpha}{2}\int (\mathbf v_\alpha-\mathbf u_\alpha)\puissance{2}
f_\alpha(\mathbf r,\mathbf v,t) \,\text{d}\puissance{3}\mathbf{v}
\end{equation}

\subsubsection{Conservation de la matière}
$$n_s=\int f_sd\mathbf{v}$$

\subsubsection{Conservation de l'impulsion}
$$n_s\mathbf{v}_s=\int \mathbf{v}_sf_sd\mathbf{v}$$
\subsubsection{Conservation de l'énergie}
\subsubsection{Conservation de la chaleur}

\subsection{Les phénomènes de transport magnétisés}
Dans un plasma collisionnel, le mouvement d'une particule peut être assimilé à
une succession de déplacements libres et de collisions, formant une marche
aléatoire de pas $\lambda_\text{lpm}$. Compte tenu du caractère isotrope du
problème, ce déplacement, sur plusieurs temps de collisions, est de
moyenne nulle. L'écart quadratique de ce déplacement à la moyenne, mesuré par la
température, ne l'est cependant pas. Il définit le processus de diffusion qui
apparaît en présence d'une inhomogénité de densité, dont la vitesse typique est
fonction de la longueur de gradient de densité et du coefficient de diffusion
$D$ :
 
\begin{equation}
    \text{V}_\text{Diff}\sim D_\alpha\frac{\nabla
    n_\alpha}{n_\alpha}=\frac{eT_\alpha}{m_\alpha\nu}\frac{\nabla
    n_\alpha}{n_\alpha}=\frac{\lambda_\text{lpm}}{L_\text{n}}v_{\text{T}_\alpha}
\end{equation}
 
où $L_\text{n}=\nabla n/n$ est la longueur de gradient de densité. 
En présence d'un champ électrique, ce déplacement aléatoire est accéléré entre
chaque collision (voir figure~\ref{1-collisions}), et entraîne un déplacement
macroscopique moyen non nul, décrit par le coefficient de mobilité :
\begin{figure}
\centering
\includegraphics[width=0.8\textwidth]{figures/collisions.jpg}
{\caption{Processus de diffusion-conduction suivant la marche aléatoire
collisionnelle d'un électron~{\cite{Rax}}.}\label{1-collisions}}
\end{figure}
\begin{equation}
    \mu_\alpha=\frac{q_\alpha}{m_\alpha\nu}
\end{equation}
 
Conduction et diffusion sont les processus de transport dominants dans les
plasmas non magnétisé. Leur coefficient de transport respectifs, $D_\alpha$ et
$\mu_\alpha$, ne sont pas indépendants et leur rapport $D_\alpha/\mu_\alpha$,
appelé énergie caractéristique, est égal à la température de l'espèce selon la relation
d'Einstein :
 
\begin{equation}
    \frac{D_\alpha}{\mu_\alpha}=T_\alpha
\end{equation}
 
\section{Modèles de plasmas magnétisés}
\subsection{Les modèles numériques}
L'étude de phénomènes complexes
%\begin{figure}[htbp]
%	\centering
%	\includegraphics[height=64mm]{figures/cave.jpg}
%	{\caption{La caverne des idées.}\label{caverne}}
%\end{figure}
%\begin{figure}[htbp]
%
%			\includegraphics[height=40mm]{figures/Magritte.jpg}
%			{\caption{Magritte. La trahison des images.}\label{magritte}}
%		\end{figure}

\subsection{Les décharges plasma basse-pression}
\label{1-transportAmbipolaire}

\subsubsection{Création de la décharge, rôle des électrons}

Dans les plasmas basse-température industriels et de laboratoires, qui
possédent un faible degré d'ionisation, ou dans l'ionospère, la dynamique du
plasma est dominée par la perte de quantité de mouvement dûe à l'ionisation la force de friction avec
le gaz.

l'établissement de la quasineutralité n'est pas instantané : à faible densité,
le champ électrique étant négligeable, les ions et les électrons diffusent
uniformément dans le volume qui les entoure.

Comme les électrons se déplacent et diffusent bien plus rapidement que les ions
du fait de leur faible masse $D\indice{e}\gg D\indice{i}$, durant un temps
$t\sim \tau\indice{e}=L\puissance{2}/D\indice{e}$, ils se dispersent jusqu'à atteindre
une paroi ou un quelconque obstable qu'ils polarisent négativement. Il apparaît
alors dans la zone proche de la surface une couche limite appelée gaine, ne
contenant que très peu d'électrons et où la quasineutralité est donc rompue. 
attirant les ions et ralentissant les électrons pour ajuster leur flux et préserver la neutralité du matériau. La zone proche de l'obstacle 



Electrical breakdown, Townsend avalanche,
\subsubsection{Transport des ions et champ ambipolaire}
\begin{equation}
\label{derivediffusion}
\end{equation}
\subsubsection{Equations de dérive-diffusion magnétisées}

\subsection{Les plasmas de bord des tokamaks}
Lawson criteriom, strongly magnetized
\subsubsection{La fusion et les tokamaks}
\subsubsection{Le plasma de la Scrape-of-Layer}
\subsubsection{Les vitesses de dérives}
\label{vitessesDerive}
%\bibliographystyle{apalike}
%\bibliography{biblio}
\end{refsection}


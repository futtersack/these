\Chapter{Introduction}
	\section{Fondamentaux de la physique des plasmas}
		 
		\emph{Le plasma}. Le plasma est souvent considéré un peu équivoquement
		comme le quatrième état de la matière. Dans notre environnement proche,
		nous avons pris conscience de son existence à travers les phénomènes de flammes,
		d'éclairs, d'aurore boréales ou d'arc électriques. Mais à des conditions de
		pressions et de températures différentes de celles de notre atmosphère
		terrestre, il est omniprésent : plus de 99\% de la matière connue est sous
		cette forme. 
		
		Une définition plus adaptée du plasma est celle d'un gaz conducteur. Une partie des
		atomes le composant est ionisée, donnant naissance à une population
		d'électrons libres et d'ions de différentes espèces. Ces populations permettent alors
		le transport de courant et, sensibles aux forces électromagnétiques, influencent
		fortement le comportement global du plasma en provoquant des phénomènes
		collectifs, non-linéaires et turbulents.
		
		Le plasma et son comportement sont décrits par la théorie de la physique des
		plasmas. Elle intègre les connaissances de nombreux domaines, tels que la
		physique statitique, l'électromagnétisme, ou encore la dynamique des fluides.
		
		\subsection{Les paramètres plasmas}
		
			Les plasmas se définissent donc comme des gaz possédant une population
			d'électrons libres $n_e$ à une température électronique $T_e$. 
			La figure \ref{zoologie}, issue du livre du National
			Research Council\cite{national1995Plasma}, représente une classification des
			plasmas en fonction de ces deux paramètres principaux qui vont influer sur la dynamique du transport de courant.
			La théorie présentée dans la suite de cette thèse ne concerne que les plasmas
			dits classiques :
			\begin{itemize}
			  \item les plasmas naturels peu dense tels que l'espace interstellaire,
			  le vent solaire, la magnétosphère, et l'ionosphère
			  \item les plasmas naturels denses tels que les éclairs et les étoiles
			  \item les plasmas industriels, de laboratoire, et thermonucléaires
			\end{itemize}
			\begin{figure}[htbp]
				\centering
				\includegraphics[height=80mm,width=64mm]{figures/zoologie.png}{\caption{Classification
				de différents plasmas en fonction de $n_e$ et $T_e$.}\label{zoologie}}
			\end{figure}
			
			Dans ces plasmas, le dégré d'ionisation $\alpha$ est donné par le rapport
			entre la densité électronique $n_e$ et la densité de gaz $n_g$ :
				$$\alpha=\frac{n_e}{n_e+n_g}$$
			Cette fraction va définir l'importance de l'interaction entre les particules 
			neutres et les particules chargées. Cependant, même à très faible $\alpha$,
			l'apparition d'une population de porteurs de charge va modifier les caractéristiques et la
			dynamique du plasma. 
			
			L'ionisation du gaz suit l'évolution de la température électronique $T_e$ qui
			mesure l'agitation thermique des électrons. Densité et température
			électronique permettent de définir le paramètre plasma:
				$$\Gamma=\frac{<E_p>}{<E_c>}=\frac{e^2n_e^{1/3}}{\epsilon_0 eT_e}$$
			Le paramètre plasma représente le ratio entre l'énergie thermique des
			électrons et leur énergie potentielle électrostatique coulombienne \emph{ie.}
			l'agitation thermique desordonnée contre les forces d'interactions
			coulombiennes structurantes. 
			
			Les plasmas classiques sont caractérisés par
			$\Gamma\ll 1$. Ils ont une population d'électrons assez espacée et/ou une
			température suffisamment élevée.
			
		\subsection{Dynamique : Echelles, équilibres et transport}
		La dynamique d'un plasma résulte du couplage entre le mouvement des
		particules chargées et les forces électromagnétiques de Coulombs ($\mathbf
		F=q_s\mathbf E$) et de Lorentz ($\mathbf F=q_s\mathbf v_s\times\mathbf B$)
		présentent ou qui se forment dans le système.
		Elle peut se décomposer sur deux échelles de temps 
			\subsubsection{Quasineutralité}
			Le processus le plus rapide est lié à
			l'équilibre microscopique électrostatique qui s'opère lors de la formation
			d'un plasma entre l'agitation thermique et l'interaction coulombienne. Les
			ions et des électrons issus de l'ionisation se réorganisent pour écranter
			leur champ électrique individuel et former ainsi un ensemble électriquement neutre. 
			Ce comportement est décrit aux plus petites échelles spatiotemporelles par
			le principe fondamental de la dynamique et la loi de Boltzman-Poisson :
			$$m_s\frac{\partial \mathbf{v}}{\partial t}=q_s\mathbf E
			\;\;\;\text{et}\;\;\;\nabla^2\Phi=\frac{\rho}{\epsilon}$$ ou
			$m_s$ est la masse de la particule, $\mathbf{v}$ sa vitesse, $q_s$ la
			charge élémentaire, $\rho$ est la différence entre la densité ionique et
			électronique.
			Les échelles fondamentales entrant en jeu sont alors la pulsation plasma et
			la longueur de Debye, reliée à la vitesse thermique 
			$$\omega_p=\sqrt{\frac{n_0e^2}{\epsilon
			m_e}}\;\text{,}\;\;\;\lambda_D=\sqrt{\frac{\epsilon
			eT_e}{n_0e^2}}\;\;\;\text{et}\;\;\;v_{th}=\sqrt{\frac{eT_e}{m_e}}$$
			Au dela de ces échelles, on peut considérer cette dynamique comme instantanée
			et uniforme. Globalement, vu de suffisement loin et exepté à ses frontières,
			le plasma est dans un état de \emph{quasi-neutralité}, \emph{ie.} $n_e=n_i$.
			
			\subsubsection{Rayon de Larmor et vitesses de dérive}
			\begin{wrapfigure}{r}{0.50\textwidth}
    			\vspace{-5pt}
    			\hspace{20pt}\includegraphics[width=0.40\textwidth]{figures/particleDrifts.png}
    			\hspace{20pt}\caption{Mouvement cyclotronique et de dérive des particules
    			dans un champ magnétique.}\label{particleDrifts}
  				 \vspace{-20pt}
			\end{wrapfigure}
			La présence d'un champ magnétique va aussi fortement influencer le transport
			du plasma. Microscpopiquement, le mouvement d'une particule chargée dans un
			champ magnétique sujette à la force de Lorentz se décrit essentiellement par trois
			composantes (cf. figure \ref{particleDrifts}) :
			
			
			\begin{itemize}
			\item Un déplacement parallèle aux lignes de champ, de l'ordre de la vitesse
			thermique des espèces, $v_\parallel\approx c_s=(eT_s/m_s)^{1/2}$
			\item Le mouvement cyclotronique, rotation rapide de la particule
			autour des lignes de champ magnétiquedans le plan perpendiculaire aux lignes
			de champ
			\item Une vitesse de dérive, 
			\end{itemize}
			\lipsum
			\subsubsection{Collisions}
			blablabla
		 Plasma parameter $\Lambda$ Debye sphere, Quasineutrality Debye length,
		Electrostatic plasma frequency $\omega_c$
		 Density : Degree of ionisation $\alpha=n_i/(n_i+n_n)$
			Temperature : Saha equation, thermal equilibrium, Maxwellian energy dist function
			Potentials : Debye sheath, Boltzmann relation
			Magnetization : Hall parameter, Larmor radius, plasma frequency
			
			
			Dans suite de cette thèse, nous allons étudier la dynamique électrique d'un plasma en milieu magnétisé
			, ie. ou les 
			des structure et de la dynamique essentiellement reliée aux variations du champ électrique.
	\section{Description fluide d'un plasma}
		
		\subsection{The Boltzmann equation}
			$$\partial_tf_s+\mathbf{v}_s\cdot\vec\nabla_\mathbf{r}f_s+\frac{\mathbf{F}_s}{m_s}\cdot\vec\nabla_{\mathbf{v}_s}f_s=\partial_tf_{|_{coll}}$$
		\subsection{Moments of the Boltzmann equation, conservation laws}
			Braginskii equations
			$$M^{(k)}_s=\int_{-\infty}^{\infty}\mathbf{v}_s^kf_sd\mathbf{v}$$
			\subsubsection{Continuity equation}
			$$n_s=\int f_sd\mathbf{v}$$
			
			\subsubsection{Momentum equation}
			$$n_s\mathbf{v}_s=\int \mathbf{v}_sf_sd\mathbf{v}$$
			\subsubsection{Energy equation}
			\lipsum{50}
			\subsubsection{Heat equation}
			\lipsum{50}
		\section{Les gaines électrostatiques}
			Les gaines électrostatiques sont des structures non-neutres qui se
			développent à la frontière entre un plasma et un objet. Elles apparaissent
			spontanément du fait de la plus grande vitesse des électrons par rapport à
			celle des ions.  d'un facteur au moins égal à même température au ratio des
			masses $\sqrt{m_i/m_e}$ Ces structures se forment ont une taille de l'ordre de grandeur de la longueur de Debye $\lambda_D$
			L'interaction dite plasma-paroi est une branche à part entière de la physique des plasmas.
			En effet, au contact d'un milieu extérieur, que ce soit un obstacle matériel
			ou un gaz ambiant, le plasma perd sa quasineutralité.
			\subsection{Physique de la prégaine}
			\subsection{Physique de la gaine}
			\subsection{Gaine dans les plasmas magnétisés}
			\subsection{Problématique de la gaine parallèle aux lignes de champ}
			\lipsum{50}

	 \section{Low-temperature plasmas}
		\subsection{Creation of the discharge, role of electrons}
		Dans les plasmas basse-température industriels et de laboratoires, qui possédent
			un faible degré d'ionisation, ou dans l'ionospère, la dynamique du plasma est dominée par
			la perte de quantité de mouvement dûe à l'ionisation la force de friction avec le gaz.
		Electrical breakdown, Townsend avalanche, 
		\lipsum{50}
		\subsection{Ions transport and ambipolar field}
	\section{Edge plasma physics of tokamaks}
		Lawson criteriom, strongly magnetized
		\subsection{Fusion and tokamaks}
		\subsection{Drift velocities}
		\subsection{Turbulence and anomalous transverse transport}
		\lipsum{50}
	

		
\Chapter{Le code MAGNIS (MAGnetized Negative Ion Source)}
\chaptermark{Modèles plasmas froids magnétisés}
Dans cette partie, nous proposons un nouveau modèle fluide pour décrire le
transport magnétisé dans les plasmas froids. Nous discutons tout d'abord de
l'approche standard de modélisation des plasmas froids, des problématiques qui
lui sont propres ainsi que des difficultés rencontrées lors du développement du
modèle basé sur les vitesses de dérive. La suite concerne l'élaboration du
nouveau modèle, nous le dérivons en expliquant les différents
termes essentiels à description du transport dans les plasmas froids puis nous
détaillons le schéma numérique original utilisé pour résoudre le modèle.
\section{Problématique}
La plupart des modèles fluides décrivant les phénomènes de transport
magnétisés dans les plasmas froids sont basés sur les équations de
dérive-diffusion (chapitre \ref{derivediffusion}). Cependant, au delà d'une
certaine intensité de champ magnétique, l'anisotropie du transport rend la résolution
numérique de ces équations complexe voire impossible. Le problème est alors
généralement adressé à travers le développement de
modèles cinétiques numériquement très coûteux, nécessitant à priori la
résolution d'échelles de l'ordre de la longueur de Debye et de la fréquence plasma.

Les équations Eqs.\ref{derivediffusion} sont en fait peu adaptées au
transport fortement magnétisé. Un indice nous est donné en considérant la
description du transport transverse par les vitesses de dérive (cf.
introduction) où l'équilibre des courants fait intervenir la divergence de
la dérive de polarisation. Cette dérive est absente du modèle stationnaire qui
néglige totalement l'inertie des particules.

Le modèle de dérive-diffusion, construit en négligeant l'inertie
des particules, ne 

\section{Dérivation du modèle}
\subsection{Hypothèses et géométrie}
\subsection{Equations de conservation}
\subsection{La gaine aux limites}
\section{Implémentation numérique du modèle}
\subsection{Source d'ionisation et pertes aux paroies}
\subsection{Inertie des particules}
\subsection{Prédictions des flux de matière et de chaleur}
\subsection{Le schéma numérique}
\subsection{Conditions aux limites}
\subsection{Mobilité et conductivité}

\section{Vérification et validation}
\subsection{Convergence en temps et en espace}
\subsection{Influence du champ magnétique}
\subsection{Réponses à la température}



\thispagestyle{empty}
\cleardoublepage 
\thispagestyle{preface}
\section*{Preface}
\addcontentsline{toc}{section}{Preface}
			\emph{La vie non examinée n'est pas digne d'être vécue }- Socrate.
			explorée physiquement et philosophiquement
			Dans la recherche de la connaissance et de la compréhension ce monde qui
		l'entoure, l'homme s'est tout d'abord intéressé à la matière qui le compose. Celle-ci
		peut nous apparaître sous trois états principaux : solide, liquide et gazeuze. 
			lorsque les atomes,structures élementaires de la matière, sont immobiles \emph{ie.} à très faible
		température, on parle d'état solide de la matière. En les laissant se déplacer
		les uns par rapport aux autres, la matière devient déformable et prend alors
		la forme d'un liquide. Enfin, une élévation de l'énergie des particules leur
		permettant une plus grande liberté, la matière se trouve à l'état gazeux, 
		tendant à occuper tout l'espace qui se trouve à sa disposition.
	
	 l'information est ce qui donne une forme à l'esprit
L'information définie par Émile Littré dans le Dictionnaire de la langue française. 
Elle se distingue de l'instruction est ce qui lui confère une structure
	Condération sur l'énergie.
	L'energie est information. Elle se propage par onde et par deplacement
	$$E=mc^2$$
	l'energie totale est une constante, elle se déplace avec le flux de chaleur
	le vide d'énergie se compose le vide absolu $m=0$
	l'energie nulle ne propage pas d'information
	à vitesse nulle, froid absolu. La masse doit être enorme pour propager d'info -> trou noir
	a vitesse tres faible les electrons doivent être extremement rapproché, fort Pe -> supraconducteur, physique stat/quantique
	une seule particule vitesse nulle masse infinie 
	bigbang trou noir->plasma solide->plasma liquide->plasma gazeux de moins en mois condcteur d'information?
	à l'inverse, une particule à qui l'on fournit de l'énergie (on l'accelere) 
	solide, liquide, gazeuze, plasma (la vitesse de l'électron est 1836*supérieure à celle de l'ion, mur du son), 
	relativiste, instantanée. 
	La vitesse te donne l'état de ta matière. L'énergie cinétique est plus grande que la potentielle
	
	le monde imaginaire de l'antimatière se propage à la seconde vitesse imaginaire solution de l'équation
	 
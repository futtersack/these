\ChapterStar{Préface}
\addstarredpart{Préface}
\markboth{Préface}{PREFACE}
\thispagestyle{preface}
\vspace{120pt}
\begin{tirade}
"L’idée que la science peut, et doit, être organisée selon des règles fixes et
universelles est à la fois utopique et pernicieuse. Elle est \emph{utopique},
car elle implique une conception trop simple des aptitudes de l’homme et des
circonstances qui encouragent, ou causent, leur développement. Et elle est
\emph{pernicieuse} en ce que la tentative d’imposer de telles règles ne peut
manquer de n’augmenter nos qualifications professionnelles qu’aux dépens de notre
humanité.
En outre, une telle idée est \emph{préjudiciable à la science}, car elle néglige
les conditions physiques et historiques complexes qui influencent en réalité le
changement scientifique. Elle rend notre science moins facilement adaptable et
plus dogmatique. (\ldots)


Le falsificationisme naïf tient ainsi pour acquis que les lois de la nature
sont manifestes, et non pas cachées sous des variations d’une ampleur
considérable ; l’empirisme considère que l’expérience des sens est un miroir du
monde plus fidèle que la pensée pure ; le rationalisme, enfin, assure que les
artifices de la raison sont plus convaincants que le libre jeu des émotions
(\ldots)

Le désir d’accroître la liberté, de mener une vie pleine et enrichissante, et
parallèlement les efforts pour découvrir les secrets de la nature et de l’homme
entraînent le rejet de tout principe universel et de toute tradition rigide.


Des études de cas comme celles des chapitres précédents (\ldots)
témoignent contre la validité universelle de n'importe quelle règle. Toutes les
méthodologies ont leur limites, et la seule "règle" qui survit, c'est : "Tout
est bon".
\end{tirade}
\hfill P.K. Feyerabend

\hfill\footnotesize  Contre la méthode, esquisse d’une théorie anarchiste de la
 connaissance (1975)
 \normalsize

\ChapterStar{}
%\addstarredpart{Préface}
%\markboth{Préface}{PRÉFACE}
	\thispagestyle{empty}
\vspace{90pt}
Les connaissances sur le sujet du transport magnétisé n'ont que très peu évolué
dans le domaine des plasmas froids depuis les années 1960 et ne suffisent plus
pour expliquer le comportement complexe du plasma que l'on
rencontre dans les sources magnétisées basse-pression actuellement en développement.
Les théories et les méthodes développées pour l'étude du transport magnétisé
dans le cadre de la recherche sur la production d'énergie par fusion
thermonucléaire sont inadaptées pour décrire la dynamique non-ambipolaire de ces
plasmas. En effet, dans ces sources, les ions ne sont que faiblement magnétisés,
les collisions avec les neutres influencent significativement le transport
tandis que les parois, puits à particules omniprésents, contrôlent les profils
d'équilibres.

Pour répondre à cette problématique, cette thèse revisite la modélisation
des plasmas froids et propose un nouveau modèle fluide décrivant le transport
magnétisé dans le plan perpendiculaire au champ magnétique.		
Nous adressons la complexité de ce transport à travers l’élaboration d'un modèle
fluide~2D\textonehalf ~et de son schéma numérique, sans approximation d'ordering
entre les longueurs caractéristiques du plasma magnétisé (i.e. la dimension du
plasma $L$, les libres parcours moyens $\lambda_{i,e}$ et les rayons
de Larmor ioniques et électroniques $\rho_{Li,e}$).
Les équations sont résolues dans le plan perpendiculaire au champ magnétique où
les asymétries et les inhomogénéités représentatives du transport magnétisé
apparaissent, tandis que les conditions aux limites (parallèles et transverses)
sont dérivées de la théorie classique de gaine. La considération de l'inertie
des particules permet de plus de capturer la dynamique transitoire du plasma
ainsi que certains types d'instabilités.
		
Le modèle, supportant une large gamme de topologies et d'intensités de champ
magnétique, est appliqué aux configurations de deux sources d'ions négatifs. 
Les asymétries et inhomogénéités observées expérimentalement sont reproduites
et, dans une géométrie représentant la Scrape-of-Layer des tokamaks, le modèle
est capable de simuler la turbulence d'interchange qui domine le transport
perpendiculaire du plasma de bord.	

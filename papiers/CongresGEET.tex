% %This is a very basic article template.
% %There is just one section and two subsections.
\documentclass[a4paper,11pt]{article} % classe "article", papier a4 et police de
% 11pt

\usepackage[utf8]{inputenc} \usepackage[T1]{fontenc}
\usepackage{cite}
\usepackage{mathptmx}

\usepackage{amsmath}
\usepackage[frenchb]{babel}
\usepackage{bm}
\usepackage{graphicx}
\usepackage{multicol}
\usepackage[runin]{abstract}
\usepackage[footnotesize,labelfont=bf,labelsep=period]{caption}
\usepackage{tocloft}
\usepackage{indentfirst}
\usepackage{sectsty}
\usepackage{calc}

\usepackage[capposition=beside,capbesideposition=right]{floatrow}

\usepackage[margin=1in]{geometry}


\title{
\vspace{-50pt}
\textbf{\Large{Modélisation fluide du transport magnétisé dans un plasma froid}}}

\author{\large{Romain FUTTERSACK}\\
\large{\emph{LAPLACE - Université Paul Sabatier Toulouse}}\\
\large{\emph{118 route de Narbonne, F-31062 Toulouse cedex 9}}}
\date{}

\begin{document}
\maketitle
\newcommand{\vecMath}[1]{\mathbf{#1}}
\renewcommand{\abstractname}{Résumé}
\abslabeldelim{:~}
\renewcommand{\abstractnamefont}{\normalsize\textbf}
\renewcommand{\abstracttextfont}{\small}

\cftpagenumbersoff{figure}
\renewcommand{\cftdotsep}{\cftnodots}
\renewcommand{\cftfigpresnum}{\textbf{FIG.~}}
\renewcommand{\cftfigaftersnum}{.}
%\cftsetindents{figure}{0em}{6em}
%\renewcommand{\cftbeforefigskip}{1em}


%\setlength{\absparindent}{0cm}
\setlength{\absparsep}{0cm}
\setlength{\absleftindent}{1.27cm}
\setlength{\absrightindent}{2.67cm}
\sectionfont{\large\textbf}

\begin{abstract}
Nous proposons un nouveau modèle fluide pour étudier la problématique du transport magnétisé dans un plasma basse pression.
Des simulations simplifiées sont analysées pour caractériser le transport dans différentes
configurations de champ magnétique représentatives de cas expérimentaux. Dans le cas de la présence d'une barrière
magnétique, le transport transverse montre une nature turbulente tandis qu'un champ magnétique homogène dans 
le plan permet un confinement radial du plasma.
\end{abstract}

\begin{section}{Introduction}
L'injection de neutres (NBI) est un système de chauffage, de génération de courant et d'alimentation pour
les machines de Fusion par Confinement Magnétique (FCM). Ce type de système est basé sur l'accélération à haute énergie 
d'un faisceau d'ions négatifs de Deutérium (D\up{-}) qui est ensuite neutralisé à travers une cible de gaz afin de 
créer un puissant faisceau de neutres D\up{0}. Le système NBI du futur réacteur ITER devra fournir une puissance de 17MW
avec des particules à 1MeV\cite{Hemsworth}. Pour atteindre ces spécifications, la source d'ions négatifs en amont du système 
doit délivrer un courant d'ions négatifs de l'ordre de 250 A.m\up{-2} uniformément distribué sur une large surface 
d'extraction ($\approx$ 0.8m\up{2}). 

La production et le maintien d'une population d'ions négatifs sont cependant difficiles et nécessitent 
l'utilisation d'un champ magnétique pour soit :
\begin{enumerate}
\item\emph{~Créer une barrière de transport électronique} 
\item\emph{~Confiner le plasma de la source de façon homogène}. 
\end{enumerate}
Cependant, la présence de ce champ magnétique dans le plasma influence fortement le déplacement des particules, 
développant d'une part de fortes inhomogénéités et favorisant d'autre part l'apparition d'instabilités pouvant 
mener à un transport de type turbulent~\cite{Zweben}.

Dans ce papier nous présentons les premières simulations obtenues à partir d'un nouveau modèle pour plasmas froids magnétisés. 
Le papier est organisé de la façon suivante : dans la première partie, nous expliquons le modèle physique développé pour 
l'étude du transport magnétisé dans les plasmas froids. Nous présentons ensuite des simulations de cas expérimentaux 
avec leurs configurations spécifiques de champ magnétique et de chauffage, \emph{ie.} en configuration filtre magnétique
dans la partie 2 et en configuration colonne de plasma confinée dans la partie 3. Finalement, une discussion des résultats
sera abordée en quatrième partie.
\end{section}
\begin{section}{MAPL2D, un code fluide pour le transport magnétisé}
Depuis quelques décennies, les plasmas basse pression sont couramment utilisés dans l'industrie et 
la recherche pour une large variété d'applications (dépôts et nettoyage de surfaces, propulseurs plasma, source d'ions
négatifs...). Conjointement avec les études expérimentales, le recours à la modélisation numérique est devenu essentiel
pour caractériser et optimiser ces dispositifs ainsi que pour mieux comprendre la physique sous-jacente de la décharge.
Le transport magnétisé est, quant à lui, un sujet qui a été et qui reste vastement étudié dans le contexte de la fusion 
contrôlée. 

MAPL2D a été conçu en réunissant l'approche usuelle de modélisation des plasmas froids et les hypothèses des plasmas 
fortement magnétisés telles que la quasineutralité ou les conditions aux limites de gaine sur les flux de particules 
et de courants dérivées du critère de Bohm. Nous conservons de plus les termes inertiels de l'équation du mouvement, 
ce qui permet de rendre compte du courant de polarisation, bien connu dans la physique de bord des tokamaks pour contrôler 
le transport transverse du potentiel électrostatique\cite{Sarazin} et qui donne lieu au mécanisme de turbulence 
d'interchange.

Nous commençons par dériver l'expression de la vitesse ionique à partir de l'équation du moment des ions :
$${\vecMath{v}_i}^{k+1}=(\nu_i^*(1+{h_i}^2))^{-1}(\vecMath{W}-h_i\vecMath{b}\times\vecMath{W}+{h_i}^2\vecMath{b}\cdot\vecMath{W}\cdot\vecMath{b})$$
avec 
$\vecMath{W}={\vecMath{v}_i}^{k}/\Delta t-e/m_i(-\vecMath{\nabla\phi}-\vecMath{\nabla P}/n)$, 
$\vecMath{b}$ le vecteur unitaire dans la direction du champ magnétique, $e/m_i$ le rapport charge sur masse ionique, 
$-\vecMath{\nabla\phi}$ le champ électrostatique dérivant du potentiel électrique, $P=n(T_e+T_i)$ la pression totale,
$\nu_i^*=\nu_i+\Delta t^{-1}$ la fréquence réduite de transfert de quantité de mouvement et $h_i=\frac{eB}{m_i\nu_i^*}$ 
le paramètre de Hall réduit. 

Le champ ambipolaire $\vecMath{E}_a$ et le courant $\vecMath{j}$ sont alors définis par :\\
$\vecMath{E}_a=-\vecMath{\nabla\phi}-\vecMath{\nabla P}/n-(m_i/e)\partial_t\vecMath{v}_i-(m_i\nu_i+m_e\nu_e)/e\vecMath{v}_i$ et 
$\vecMath{j}=(\nu_e^*(1+{h_e}^2))^{-1}(\vecMath{H}+h_e\vecMath{b}\times\vecMath{H}+{h_e}^2\vecMath{b}\cdot\vecMath{H}\cdot\vecMath{b})$
avec $\vecMath{H}=e^2n(-\vecMath{\nabla\phi}-\vecMath{E}_a)/m_e+n^{k+1}\vecMath{j}/(n^k\Delta t)$.\\

En utilisant ces définitions, on peut écrire l'équation de conservation de la matière :
\begin{equation}
\label{density}\partial_t n+\vecMath{\nabla}\cdot(n\vecMath{v}_i)=Sn
\end{equation}
tandis que l'équation de conservation du courant $\nabla\cdot(\vecMath{j})=0$ nous permet d'obtenir l'équation sur le 
potentiel électrostatique $\phi$:
\begin{equation}
\label{current}\vecMath{\nabla}\cdot\left(-(\nu_e^*(1+{h_e}^2))^{-1}(\vecMath{\nabla\phi}+h_e\vecMath{b}\times\vecMath{\nabla\phi})\right)=
\vecMath{\nabla}\cdot\left((\nu_e^*(1+{h_e}^2))^{-1}(\vecMath{G}+h_e\vecMath{b}\times\vecMath{G})\right)
\end{equation}
où $\vecMath{G}=e^2n\vecMath{E}_a/m_e-n^{k+1}\vecMath{j}/(n^k\Delta t)$. 
$$\vecMath{\nabla}\cdot(-\frac{\vecMath{\nabla\phi}+h_e\vecMath{b}\times\vecMath{\nabla\phi}}{\nu_e^*(1+{h_e}^2)})=
\vecMath{\nabla}\cdot(\frac{\vecMath{G}+h_e\vecMath{b}\times\vecMath{G}}{\nu_e^*(1+{h_e}^2)})$$
Les équations (\ref{density}) et (\ref{current}) contrôlent l'évolution temporelle de la densité et du potentiel 
électrostatique.
\end{section}

\begin{section}{Un transport anormal à travers la barrière magnétique}
\begin{figure}
\capbeside
\begin{floatrow}[2]
	\ffigbox[\hsize]
	{\includegraphics[height=48mm,width=60mm]{figures/magneticBarrier.png}}{\caption{Carte de densité dans la configuration 'Barrière magnétique'.}\label{figMagneticBarrier}}
	\hfill
	\ffigbox[\hsize]
	{\includegraphics[height=45mm,width=60mm]{figures/profileDensite.png}}{\caption{Profile axial moyen de densité avec et sans turbulence.}\label{figProfileDensite}}
\end{floatrow}
\end{figure}
La première configuration de champ magnétique étudiée est utilisée dans le concept de source d'ions négatifs modulaire 
retenu pour ITER. La source, de dimension 20~cm par 20~cm, est décomposée en trois régions : la zone driver-chambre 
d'expansion où le plasma est généré et diffuse dans le volume ; la zone d'extraction, où les ions négatifs sont créés sur
la paroi de césium puis extraits à travers une grille polarisée, se situe à de l'autre côté de la source ; enfin, 
la problématique zone de filtre magnétique, utilisée pour diminuer la température électronique et ainsi réduire la 
perte des ions négatifs par détachement collisionnel. La figure \ref{figMagneticBarrier} représente le plan médian du module, 
perpendiculaire au champ magnétique. L'amplitude du champ magnétique suit une gaussienne centrée au milieu du domaine et de 
2 cm de largeur, formant une barrière de transport pour les électrons (voir fig. \ref{figProfileDensite}). 
Sur cette même figure, on peut voir que les résultats des simulations montrent un accord qualitatif avec les profils d'équilibre
des modèles stationnaires. L'évolution du plasma est de plus cohérente avec de récents résultats obtenus à l'aide d'un modèle 
PIC\cite{Boeuf} : Le flux électronique suit le courant diamagnétique $\vecMath{J}_{dia}=\frac{\nabla P\times\vecMath{B}}{B^2}$
dû au gradient de pression, remontant le long de la barrière jusqu'à proximité de la paroi, où le fort champ électrique 
généré par la gaine lui permet de traverser la barrière par la dérive $\vecMath{E}\times\vecMath{B}$, entraînant la forte
inhomogénéité de densité près de la grille d'extraction. Sur la figure \ref{figMagneticBarrier}, on peut voir le développement
d'une instabilité de même longueur d'onde et de même fréquence que celle observée avec le code PIC. Celle-ci 
génère une dynamique de style avalanche, caractérisée par la relaxation du profil de densité et par de forts flux intermittents de 
particules~\cite{Sarazin} laissant apparaître la nature turbulente du transport dans les plasmas magnétisés. 
Le transport à travers la barrière magnétique est alors fortement amplifié.

\end{section}

\begin{section}{Confinement de la colonne de plasma dans un champ magnétique uniforme}
La configuration précédente entraînant une forte inhomogénéité verticale du plasma au niveau de la zone d'extraction, 
Cybele, un autre concept de source d'ions négatifs de forme rectangulaire et fine (15~cm x 20~cm x 1.2~m), est à 
l'étude~\cite{Simonin}. Le plasma, généré au centre de la source, est créé directement dans un champ magnétique vertical uniforme confinant. 
Dans le plan transverse au champ magnétique, la vitesse de dérive met le plasma en rotation, formant ainsi une longue 
colonne de plasma homogène suivant l'axe vertical de la source. Dans cette géométrie, les simulations issues de MAPL2D 
confortent la théorie sur l'évolution du plasma : une rotation a lieu dans le plan perpendiculaire au champ magnétique 
et le plasma est bien confiné radialement. Les résultats du code seront bientôt confrontés aux mesures expérimentales et aux
résultats du code PIC, sur les profils mesurés et sur les caractéristiques du transport turbulent.
\end{section}

\begin{section}{Conclusions} 
Dans ce travail, nous avons présenté un nouveau modèle de plasma basse-pression magnétisé, basé sur l'approche 
usuelle de modélisation des plasmas froids ainsi que sur l'adaptation de méthodes développées pour les plasmas de 
bord des tokamaks. 

Nous avons aussi testé MAPL2D sur deux géométries de champ magnétique représentatifs de dispositifs expérimentaux 
et comparé les résultats des simulations avec un code PIC en trouvant un bon accord qualitatif. Dans la configuration 
filtre magnétique, les profils d'équilibres de densité et de potentiel électrostatique sont similaires. Le transport
à travers la barrière magnétique montre une nature turbulente avec des caractéristiques du même ordre de grandeur que
celles obtenues avec le code PIC. Dans la configuration colonne de plasma, les premières simulations sont cohérentes avec
les résultats théoriques attendus, en attendant d'être confrontés aux mesures expérimentales et aux autres modèles.
\end{section}

\section*{Remerciements}
Ce travail, supporté par la Communauté Européenne par l'association entre EURATOM et le CEA, a été mené dans le contexte de 
l'European Fusion Development Agreement et a été financièrement supporté par l'Agence Nationale de la Recherche (ANR) au sein
des projets ITER-NIS (BLAN08-2\_310122) and METRIS (ANR-11-JS09-008). Les avis et opinions exprimées dans ce document ne 
reflètent pas nécessairement ceux de la Commission européenne.

\begin{thebibliography}{1}

\bibitem{Hemsworth} R.S. Hemsworth et al, {\em Rev. Sci. Instrum 79} (2008), 02C109
\bibitem{Zweben} S. J. Zweben et al, {\em Plasma Phys. Control. Fusion 49} (2007), S1
\bibitem{Sarazin} Y. Sarazin, Ph. Ghendrih, {\em Phys. Plasma 5} (1998), 4214
\bibitem{Boeuf} St. Kolev et al, {\em Plasma Sources Sci. Technol. 21} (2012), 025002
\bibitem{Simonin} A. Simonin et al, {\em Nucl. Fusion 52} (2012), 063003

\end{thebibliography}
\end{document}

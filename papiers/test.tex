% %This is a very basic article template.
% %There is just one section and two subsections.
\documentclass[a4paper,12pt]{article} % classe "article", papier a4 et police de
% 11pt

\usepackage[utf8]{inputenc} \usepackage[T1]{fontenc}
\usepackage{mathptmx}
\usepackage[english]{babel}
\usepackage{amsmath,amsfonts,amsthm,amssymb}
\usepackage{graphicx}
\usepackage{multicol}
\usepackage{abstract}

% \usepackage{titlepages}

\usepackage[margin=25mm]{geometry}

\begin{document}
\linespread{1}

\thispagestyle{empty}
\linespread{1.6}
\begin{abstract}

While various low-pressure plasma sources operating with a steady magnetic field are widely used in industrial and research 
applications, the knowledge of magnetised transport in these non-thermal plasmas has hardly evolved since the 60s. 
As the transport of charges and currents in such plasma sources may show a complex and ill-unsderstood
behavior, we investigate the issue of magnetised transport as such by using a new 2D fluid 
model, combining the usual methods of cold plasma modeling with modeling techniques from hot fusion plasmas research, 
allowing to explore a large range of magnetic field strenghs and topologies. We then analyse simulations related 
to representative experiments with various magnetic field configurations in order to characterize 
the transport in these low-temperature plasmas and compare the results with experimental data and application-oriented 
models. 
In the ITER negative ion source, the plasma develops asymetry due to the presence of the magnetic filter and the tranverse 
transport may exhibit some turbulence. In the long and thin plasma source CYBELE, the homogeneous magnetic field is found
to improve the radial confinement of the plasma, reducing the particles losses to the wall. Finaly the model is able to
reproduce the intermittent turbulent transport which takes place in the highly magnetised Scrape-of-Layer of tokamaks.
\end{abstract}

%%%%%%%%%%%%%%%%%%%%%%%%%%%%%%%%%%%%%%%%%%%%%%%%%%%%%%%%%%
%%
%%  PROJECT: Thèse RFU
%%
%%  Created by Romain Futtersack on 20/01/11.
%%  Copyright 2011 __UPS-CEA__. 
%%  
%%  All rights reserved.
%%
%%%%%%%%%%%%%%%%%%%%%%%%%%%%%%%%%%%%%%%%%%%%%%%%%%%%%%%%%%

\documentclass[12pt]{article}

%%%%%%%%%%%%%%%%%%%%%%%%%%%%%%%%%%%%%%%%%%%%%%%%%%%%%%%%%%
%% 
%% MARK: Packages declarations
%%
%%%%%%%%%%%%%%%%%%%%%%%%%%%%%%%%%%%%%%%%%%%%%%%%%%%%%%%%%%



%%%%%%%%%%%%%%%%%%%%%%%%%%%%%%%%%%%%%%%%%%%%%%%%%%%%%%%%%%
%% 
%% MARK: Fonts declarations
%%
%%%%%%%%%%%%%%%%%%%%%%%%%%%%%%%%%%%%%%%%%%%%%%%%%%%%%%%%%%

%\usepackage[utf8]{inputenc}
\usepackage[utf8]{inputenc} \usepackage[T1]{fontenc}
\usepackage{mathptmx}
\usepackage[english]{babel}
\usepackage{amsmath,amsfonts,amsthm,amssymb}
\usepackage{graphicx}
\usepackage{multicol}
\usepackage{abstract}
\usepackage[center,labelfont=bf]{caption}
\usepackage{tocloft}
\usepackage{indentfirst}

\usepackage[margin=25mm]{geometry}


%%%%%%%%%%%%%%%%%%%%%%%%%%%%%%%%%%%%%%%%%%%%%%%%%%%%%%%%%%
%% 
%% MARK: Input macro file
%% 
%%%%%%%%%%%%%%%%%%%%%%%%%%%%%%%%%%%%%%%%%%%%%%%%%%%%%%%%%%



%%%%%%%%%%%%%%%%%%%%%%%%%%%%%%%%%%%%%%%%%%%%%%%%%%%%%%%%%%
%% 
%% Begin Document
%% 
%%%%%%%%%%%%%%%%%%%%%%%%%%%%%%%%%%%%%%%%%%%%%%%%%%%%%%%%%%

\begin{document}

	%%%%%%%%%%%%%%%%%%%%%%%%%%%%%%%%%%%%%%%%%%%%%%%%%%%%%%%%%%
	%% 
	%% MARK: Title & Data
	%% 
	%%%%%%%%%%%%%%%%%%%%%%%%%%%%%%%%%%%%%%%%%%%%%%%%%%%%%%%%%%


\title{Modèle plasma froid magnétisé}
\author{\small{R. Futtersack\textsuperscript{1,2},
		P. Tamain\textsuperscript{1}, 
		G. Hagelaar\textsuperscript{2}, 
		Ph. Ghendrih\textsuperscript{1},}\\ 
		\small{J.P. Boeuf\textsuperscript{2}, 
		A. Simonin\textsuperscript{1}}\\
		\scriptsize{\textit{\textsuperscript{1}CEA, IRFM, F-13108 St. Paul-lez-Durance, France}}\\
		\scriptsize{\textit{\textsuperscript{2}Universite Paul Sabatier Toulouse - LAPLACE - 118 route de Narbonne, F-31062 Toulouse cedex 9}}}
\date{}
%%\maketitle

\thispagestyle{empty}

\newpage
\section*{Modèle plasma froid / magnétisé}

On commence avec l'équation du mouvement
\begin{align}
\label{eqMov}&-\vec{\nabla}p+qn\vec{E}+qn\vec{v}\wedge\vec{B}-\frac{1}{\mu}n\vec{v}=\vec{0}
\end{align}
où $p$ est la pression, $q$ la charge, $n$ la densité plasma, $E$ le champ électrique, $v$ la vitesse, $B$ le champ magnétique
et $\mu=\frac{q}{m\nu}$ la mobilité, avec $\nu$ une fréquence d'ionisation / collision (?).
\\
En prenant successivement le produit vectoriel et produit scalaire de $\vec{b}$ avec (\ref{eqMov}) 
pour obtenir les projections perpendiculaires et parallèles, on obtient :
\begin{align}
\label{eqMovCrossB}&-\vec{b}\wedge\vec{\nabla}p+qn\vec{b}\wedge\vec{E}+qn\vec{v}B+\frac{1}{\mu}n\vec{v}\wedge\vec{b}=\vec{0}\\
\label{eqMovDotB}&-\vec{\nabla}_{\parallel}p+qn\vec{E}_{\parallel}-\frac{1}{\mu}n\vec{v}_{\parallel}=\vec{0}
\end{align}
\\
Soit la fréquence cyclotronique $\omega_c=\frac{qB}{m}$, on peut définir le paramètre de Hall 
$h=\mu B=\frac{\omega_c}{\nu}$. On élimine le terme en $\vec{v}\wedge\vec{B}$ par 
(\ref{eqMov}) - $h$(\ref{eqMovCrossB}) + $h^2$(\ref{eqMovDotB}) :
\begin{align}
&-\vec{\nabla}p-h^2\vec{\nabla}_{\parallel}p+qn\left(\vec{E}+h^2\vec{E}_{\parallel}\right)+h\vec{b}\wedge\vec{\nabla}p\\
&=\frac{1}{\mu}n\vec{v}\left(1+h^2)\right)\\
\end{align}












\end{document}
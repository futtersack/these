%%%%%%%%%%%%%%%%%%%%%%%%%%%%%%%%%%%%%%%%%%%%%%%%%%%%%%%%%%
%%
%%  PROJECT: Thèse RFU
%%
%%  Created by Romain Futtersack on 20/01/11.
%%  Copyright 2011 __UPS-CEA__. 
%%  
%%  All rights reserved.
%%
%%%%%%%%%%%%%%%%%%%%%%%%%%%%%%%%%%%%%%%%%%%%%%%%%%%%%%%%%%

\documentclass[12pt]{article}

%%%%%%%%%%%%%%%%%%%%%%%%%%%%%%%%%%%%%%%%%%%%%%%%%%%%%%%%%%
%% 
%% MARK: Packages declarations
%%
%%%%%%%%%%%%%%%%%%%%%%%%%%%%%%%%%%%%%%%%%%%%%%%%%%%%%%%%%%

\usepackage{amssymb}
\usepackage{amscd}
\usepackage{xcolor,graphicx}
\usepackage{verbatim}

\usepackage{amsmath,amsthm} 
\usepackage{enumerate}

%%%%%%%%%%%%%%%%%%%%%%%%%%%%%%%%%%%%%%%%%%%%%%%%%%%%%%%%%%
%% 
%% MARK: Fonts declarations
%%
%%%%%%%%%%%%%%%%%%%%%%%%%%%%%%%%%%%%%%%%%%%%%%%%%%%%%%%%%%

\usepackage{eucal}
\usepackage{setspace} 
\usepackage{multicol}
\usepackage{multirow}
\usepackage{hhline}
\usepackage{longtable}
\usepackage{array}
\usepackage{placeins}
\usepackage{hyperref}
\usepackage[francais]{layout} 
\usepackage{pstricks,pstricks-add,pst-math,pst-xkey}
\usepackage[frenchb]{babel}
\usepackage[utf8]{inputenc}
\usepackage[T1]{fontenc}


%%%%%%%%%%%%%%%%%%%%%%%%%%%%%%%%%%%%%%%%%%%%%%%%%%%%%%%%%%
%% 
%% MARK: Input macro file
%% 
%%%%%%%%%%%%%%%%%%%%%%%%%%%%%%%%%%%%%%%%%%%%%%%%%%%%%%%%%%

%%
%%  MACROS FOR Th_se_RFU
%%
%%  Created by Romain Futtersack on 20/01/11.
%%  Copyright 2011 __MyCompanyName__. All rights reserved.
%%

%%%%%%%%%%%%%%%%%%%%%%%%%%%%%%%%%%%%%%%%%%%%%%%%%%%%%%%%%%
%% 
%% MARK: Page Display
%%
%%%%%%%%%%%%%%%%%%%%%%%%%%%%%%%%%%%%%%%%%%%%%%%%%%%%%%%%%%

%%%%%%%%%%%%%%%%%%%%%%%%%%%%%%%%%%%%%%%%%%%%%%%%%%%%%%%%%%
%% 
%% MARK: Packages declarations
%%
%%%%%%%%%%%%%%%%%%%%%%%%%%%%%%%%%%%%%%%%%%%%%%%%%%%%%%%%%%
\usepackage[utf8]{inputenc} 
\usepackage[T1]{fontenc}
\usepackage{xcolor,graphicx}
\usepackage[a4paper]{geometry}
\setlength{\textheight}{675pt}
\setlength{\topmargin}{-32pt}
\setlength{\footskip}{60pt}
\DeclareUnicodeCharacter{00A0}{~}
\usepackage{wrapfig}
%\usepackage{verbatim}
\usepackage{enumerate}
\usepackage{subfloat}

\usepackage{textcomp}
\usepackage{amsmath}


\usepackage{libertine}
%\usepackage{palatino}
%\usepackage[eulergreek]{mathastext}
\pdfinclusioncopyfonts=1
\usepackage[french]{babel}


\usepackage[center,labelfont=bf]{caption}
\usepackage{indentfirst}
\usepackage{floatrow}
%\usepackage[margin=25mm]{geometry}
\usepackage[subfigure]{tocloft} 
\uchyph=0
\interfootnotelinepenalty=10000
%\renewcommand{\cftbeforetoctitleskip}{-0.25in}        % Title is 1in from top
% title
\renewcommand{\cftchapnumwidth}{8mm}
\renewcommand{\cftsecnumwidth}{4mm}
\renewcommand{\cftsubsecnumwidth}{8mm}
\renewcommand{\cfttoctitlefont}{\hfill \huge  \bfseries}
\renewcommand{\cftaftertoctitle}{\hfill}

%Une table des matiere par chapitre

\usepackage{chngcntr}
\numberwithin{equation}{section}
\usepackage[squaren,Gray,thinspace]{SIunits}
\usepackage{lipsum}

%ajoute la bibliographie dans la table des matieres
\usepackage[nottoc]{tocbibind}
\renewcommand{\tocbibname}{Liste des références}


\setlength{\cftbeforesecskip}{0.5em}
\renewcommand\cftchapfont{\large\bfseries}
\renewcommand\cftsecfont{\normalsize\bfseries}
\renewcommand{\cftpartleader}{\cftdotfill{\cftdotsep}}
\usepackage{fancyhdr}


\setlength{\headheight}{30pt}
\fancypagestyle{preface}{%
    \fancyhead{}
	\fancyfoot{}
	\renewcommand{\headrulewidth}{0pt}
	\fancyfoot[C]{\thepage}
}
\fancypagestyle{chapitre}{%
    \fancyhead{}
	\fancyfoot{}
	\renewcommand{\headrulewidth}{0.4pt}
	% Increase the length of the header such that the folios
    % (typography jargon for page numbers) move into the margin
    \fancyhfoffset[LE]{6mm}% slightly less than 0.25in
    \fancyhfoffset[RO]{6mm}%
    
	% Put some space and a vertical bar between the folio and the rest of the header
    \fancyhead[LE]{\rule[-2ex]{0pt}{2ex}\small{\thepage}\hfill\rightmark}%
    \fancyhead[RO]{\rule[-2ex]{0pt}{2ex}\leftmark\hfill\small{\thepage}}%
	\fancyfoot[C]{\thepage}
}
\renewcommand{\cftpartfont}{\large\bfseries} % make parts look like sections in TOC
\renewcommand{\cftpartpagefont}{\normalfont}
\setlength{\cftbeforepartskip}{10pt}
\usepackage{ifthen}
\newcommand{\Chapter}[1]{
	\ifthenelse{\isodd{\thepage}}{\newpage\thispagestyle{empty}~}{}
	\newpage
	\pagestyle{empty}
	\cleardoublepage
	\titleformat{\chapter}[display]
		{\normalfont\huge\bfseries}{\chaptertitlename\ \thechapter}{20pt}{\Huge}
	\chapter{#1}
	\nomtcrule
	\minitoc
	\newpage
	\thispagestyle{empty}
	\cleardoublepage
	\pagestyle{chapitre}}
\newcommand{\ChapterStar}[1]{
	\ifthenelse{\isodd{\thepage}}{\newpage\thispagestyle{empty}~}{}
	\newpage
	\pagestyle{empty}
	\cleardoublepage
	\titleformat{\chapter}[display]
		{\normalfont\huge\bfseries}{\chaptertitlename\
		\thechapter}{20pt}{\vspace{-80pt}\Huge} \chapter*{#1}
	\pagestyle{chapitre}}
\usepackage{appendix} % Makes appendices
\usepackage{cclicenses} % Creative commons
\usepackage{afterpage} % Makes appendices
\usepackage{microtype} % makes pdf look better


\usepackage{setspace}

\renewcommand{\theequation}{\thechapter-\thesection.\arabic{equation}}
\renewcommand{\thechapter}{\Roman{chapter}}
\renewcommand{\thesection}{\arabic{section}}

\parindent=1em
\parskip 0.5em plus 1pt
\newcommand{\Section}[1]{\newpage\section{#1} \setcounter{figure}{0}
\setcounter{equation}{0}}
\renewcommand{\thefigure}{\arabic{section}.\alph{figure}}
\counterwithin{figure}{section}

\makeatletter
\renewcommand{\fnum@figure}{\small \textsc{\figurename}~\thefigure\it}
\renewcommand{\fnum@table}{\small \textsc{\tablename}~\thetable\it}

\makeatother

\newcommand{\puissance}[1]{^{{#1}}}
%\newcommand{\indice}[1]{_{_{#1}}}
\newcommand{\indice}[1]{{_{#1}}}

\newcommand{\para}{{\mkern3mu\vphantom{\perp}\vrule depth 0pt\mkern3mu\vrule depth 0pt\mkern3mu}}
\renewcommand{\eqref}[1]{\hbox{Eq.~\ref{#1}}}
\newcommand{\eqrefp}[1]{\hbox{(Eq.~\ref{#1})}}

\newcommand{\overbar}[1]{\mkern 1.5mu\overline{\mkern-1.5mu#1\mkern-1.5mu}\mkern 1.5mu}
\setlength{\footnotesep}{0.5cm}
\setlength{\skip\footins}{1cm}

%\newcommand{\artlabel}[1]{{\sc #1.}}
\usepackage{titlesec}

\usepackage[final]{pdfpages} 
\usepackage{nameref}

\usepackage[refsection=chapter, isbn=false,bibencoding=utf8,
    url=false,sorting=none,sortcites=true,
    doi=false,
    eprint=false,
backend=biber]{biblatex}

\addbibresource{biblio.bib}
\usepackage{subfigure}

\usepackage{epstopdf} 

\usepackage{booktabs}
\newcommand{\ra}[1]{\renewcommand{\arraystretch}{#1}}
\usepackage{multicol}
\usepackage{csquotes}
\usepackage{cancel}
\newcommand\Ccancel[2][black]{\renewcommand\CancelColor{\color{#1}}\cancel{#2}}
\newenvironment{tirade} 
   {\begin{quotation} \itshape\small} 
   {\normalsize \end{quotation}} 

\usepackage{nomencl,etoolbox,ragged2e}
\usepackage{mhchem}
\renewcommand\nomgroup[1]{%
  \item
  \item[\large\bfseries
  \ifstrequal{#1}{C}{Constantes}{%
  \ifstrequal{#1}{R}{Symboles romans}{%
  \ifstrequal{#1}{G}{Symboles grecs}{%  
  \ifstrequal{#1}{A}{Acronymes}{}}}}]
  }

\renewcommand*{\nompreamble}{\markboth{\nomname}{\nomname}}

\newcommand{\nomdescr}[1]{\parbox[t]{8cm}{\RaggedRight #1}}
\newcommand{\nomwithdim}[5]{\nomenclature[#1]{#2}%
{\nomdescr{#3}\parbox[t]{3cm}{\RaggedRight #4}\parbox[t]{2cm}{\RaggedLeft #5}}}

\makenomenclature



%\bibliography{biblio}
%\uspackage[authoryear]{natbib}
%\usepackage{chapterbib}
%\bibliographystyle{siam}
\definecolor{perso}{rgb}{0,0,0.8}
\usepackage{hyperref}
\hypersetup{
     unicode=true,          % non-Latin characters in Acrobat’s bookmarks
     pdftoolbar=true,        % show Acrobat’s toolbar?
     pdfmenubar=true,        % show Acrobat’s menu?
     pdffitwindow=false,     % window fit to page when opened
     pdfstartview={FitH},    % fits the width of the page to the window
     pdftitle={These Futtersack},% title
     pdfauthor={Romain Futtersack},     % author
     pdfsubject={Fluid Modeling of Magnetized Transport in Low-Temperature
     Plasmas},   % subject of the document
     pdfcreator={Romain Futtersack},   % creator of the document
     pdfproducer={LAPLACE}, % producer of the document
     pdfkeywords={Fluid} {Modeling} {Magnetized} {Low-Temperature} {Plasmas}
     {Transport}, % list of keywords
     pdfnewwindow=true,      % links in new window
     colorlinks=true,       % false: boxed links; true: colored links
     linkcolor=perso,          % color of internal links (change box color with
     % linkbordercolor)
     citecolor=perso,        % color of links to bibliography
     filecolor=magenta,      % color of file links
     urlcolor=cyan           % color of external links
 }

\usepackage[french]{minitoc}
\setcounter{minitocdepth}{2}
\renewcommand{\mtctitle}{}

\dominitoc


%%%%%%%%%%%%%%%%%%%%%%%%%%%%%%%%%%%%%%%%%%%%%%%%%%%%%%%%%%
%% 
%% Begin Document
%% 
%%%%%%%%%%%%%%%%%%%%%%%%%%%%%%%%%%%%%%%%%%%%%%%%%%%%%%%%%%

\begin{document}

	%%%%%%%%%%%%%%%%%%%%%%%%%%%%%%%%%%%%%%%%%%%%%%%%%%%%%%%%%%
	%% 
	%% MARK: Title & Data
	%% 
	%%%%%%%%%%%%%%%%%%%%%%%%%%%%%%%%%%%%%%%%%%%%%%%%%%%%%%%%%%
	
	
	\renewcommand{\labelitemi}{$\bullet$}
	\begin{titlepage}
		\vspace*{-110pt}
		%\hspace*{-20pt}\includegraphics[height = 100pt]{irfm}\hfill\includegraphics[height = 100pt]{ecp}\hspace*{-20pt}
		\vspace*{150pt}
		\hrule
		\begin{center}
			\textsc{\textbf{
			\begin{spacing}{0.9}
				\Huge Modélisation du transport des électrons à travers 
					une barrière magnétique dans un plasma froid
			\end{spacing}}
			\large Romain Futtersack}
			\normalsize\today
		\end{center}
		\hrule
		\vfill
		
		\hspace*{-40pt}
		\begin{minipage}{200pt}\begin{flushleft}
			\emph{Tutrice : Mireille Schneider}
			
			CEA/DSM/IRFM/SCCP/GSEM
			
			CEA Cadarache
			
			13108 St Paul Lez Durance Cedex
			
			France
			
			+33.442.25.62.15
		\end{flushleft}
		\end{minipage}
		\begin{minipage}{240pt}\begin{flushright}
			\emph{Suiveur : Florian DeVuyst}
			
			Ecole Centrale Paris - LMAS
			
			Grandes voies des Vignes
			
			92295 Châtenay-Malabry Cedex
			
			France
			
			+33.141.13.17.19
		\end{flushright}
		\end{minipage}
		\vspace{-80pt}
	\end{titlepage}
	\newpage
	%%%%%%%%%%%%%%%%%%%%%%%%%%%%%%%%%%%%%%%%%%%%%%%%%%%%%%%%%%
	%% 
	%% MARK: Abstract
	%% 
	%%%%%%%%%%%%%%%%%%%%%%%%%%%%%%%%%%%%%%%%%%%%%%%%%%%%%%%%%%
	\thispagestyle{empty} 
	~\newpage
	\setcounter{page}{1}
	\pagenumbering{roman}
	\vspace*{-110pt}
	\renewcommand{\abstractname}{Abstract}
	\begin{abstract}
		
	\end{abstract}
	\vfill
	\renewcommand{\abstractname}{Résumé}
	\begin{abstract}
		Dans le bord d'un plasma de tokamak (Scrape Of Layer) 
		ou dans la chambre d'expansion d'une source d'ions négatifs,
		le transport des particules est régit par la physique des
		lignes de champ ouvertes et par la théorie de gaine.
		Ce travail vise à étudier ce transport à l'aide d'un code 
		de simulation 3D.
	\end{abstract}
	\vfill
	\newpage
	%%%%%%%%%%%%%%%%%%%%%%%%%%%%%%%%%%%%%%%%%%%%%%%%%%%%%%%%%%
	%% 
	%% MARK: Acknowledgement
	%% 
	%%%%%%%%%%%%%%%%%%%%%%%%%%%%%%%%%%%%%%%%%%%%%%%%%%%%%%%%%%
	\thispagestyle{empty}
	~\newpage
	\setcounter{page}{2}
	\vfill
	\vspace*{20pt}
	\section*{\center Remerciements}
	
	~\\
	\vfill
	\newpage
	
	%%%%%%%%%%%%%%%%%%%%%%%%%%%%%%%%%%%%%%%%%%%%%%%%%%%%%%%%%%
	%% 
	%% MARK: Table of contents
	%% 
	%%%%%%%%%%%%%%%%%%%%%%%%%%%%%%%%%%%%%%%%%%%%%%%%%%%%%%%%%%
	\thispagestyle{empty} 
	~\newpage
	\tableofcontents
	\thispagestyle{empty} 
	\setcounter{page}{0}
	\newpage
	
	
	%%%%%%%%%%%%%%%%%%%%%%%%%%%%%%%%%%%%%%%%%%%%%%%%%%%%%%%%%%
	%% 
	%% MARK: Introduction
	%% 
	%%%%%%%%%%%%%%%%%%%%%%%%%%%%%%%%%%%%%%%%%%%%%%%%%%%%%%%%%%
	\pagenumbering{arabic}
	\section*{Introduction}
	
	\begin{header}
		Some header text.    
	\end{header}	
	
	%%%%%%%%%%%%%%%%%%%%%%%%%%%%%%%%%%%%%%%%%%%%%%%%%%%%%%%%%%
	%% 
	%% MARK: The sections
	%% 
	%%%%%%%%%%%%%%%%%%%%%%%%%%%%%%%%%%%%%%%%%%%%%%%%%%%%%%%%%%
	\newpage
	
\section{Physique de la turbulence de bord}
\subsection{Moments cinetiques}
Equation cinetique de Vlasov 
$$\partial_t f_s
+\vec{v}_s\cdot\nabla_{\vec{r}}f_s+\frac{\vec{F}_s}{m_s}\cdot\nabla_{\vec{v}}f_s=\partial_tf{|_{coll}}$$
Comme $\nabla_{\vec{v}}\cdot\vec{F}_s=0$ on peut l'ecrire sous sa forme
conservative :
$$\partial_t f_s
+\nabla_{\vec{r}}\left(f_s\vec{v}_s\right)+\nabla_{\vec{v}}\left(\frac{\vec{F}_s}{m_s}f_s\right)=C$$
\subsubsection{Equation de conservation de la matiere}
$$\partial_t\int_v f_s d^3v
+\int_v\vec{v}_s\cdot\nabla_{\vec{r}}f_sd^3v+\int_v\frac{\vec{F}_s}{m_s}\cdot\nabla_{\vec{v}}f_sd^3v=\int_vCd^3v$$


$$L_\perp\gg \rho^i_{l}$$
$$\epsilon\equiv\frac{\omega}{\omega_{ci}}\ll 1$$
$$\partial_tn+\nabla\left(n\vec{u}\right)=0$$
$$nm\left(\partial_t\vec{u}+\vec{u}\cdot\nabla\vec{u}\right)=nq\left(\vec{E}+\vec{u}\times\vec{B}\right)-\nabla
P$$
$$nm\left(\partial_t\vec{u}_\perp+\vec{u}_\perp\cdot\nabla\vec{u}\right)=nq\left(\vec{E}_\perp+\vec{u}_\perp\times\vec{B}\right)-\nabla_\perp
P$$
$$\vec{u}^{(1)}_\perp\times\vec{B}=\frac{1}{nq}\nabla_\perp P-\vec{E}_\perp$$

$$\vec{u}_E=\frac{\vec{B}\times\nabla_\perp U}{B^2}$$
$$\vec{u}_{dia}=\frac{\vec{B}\times\nabla_\perp P}{nqB^2}$$
$$\vec{u}^{(2)}_\perp=\vec{u}_{pol}=-\frac{m}{qB^2}\left(\frac{d\nabla_\perp U}{dt}\right)_\perp$$

$$\partial_tn+\nabla_\perp\left(n\vec{u}_\perp\right)+\nabla_\parallel\left(n\vec{u}_\parallel\right)=0$$
$$n\nabla_\perp\vec{u}_E=nB\left[\frac{1}{B^2},U\right]$$
$$\vec{u}_E\nabla_\perp n=\frac{\vec{B}\times\nabla_\perp
U}{B^2}\nabla_\perp\left(n\right)=\vec{B}\frac{\nabla_\perp
U\times\nabla_\perp n}{B^2}=\frac{1}{B}\left[U,n\right]$$
$$\nabla_\perp\left(n\vec{u}_{dia}\right)=\nabla_\perp\frac{\vec{B}\times\nabla_\perp
P}{qB^2}=\frac{1}{q}B\left[P,\frac{1}{B^2}\right]=\frac{1}{q}BT\left[n,\frac{1}{B^2}\right]$$
$$\nabla_\parallel\left(n\vec{u}_\parallel\right)=B\nabla_\parallel\frac{j^e_\parallel}{B}=\sigma_\parallel
N \exp\left(\Lambda-\Phi\right)$$
$$\nabla_\perp B \ll \nabla_\perp n $$
$$\partial_tN+\nabla_\perp\left(N \vec{u}_E\right)=\sigma_\parallel N \exp\left(\Lambda-\Phi\right)+D\nabla^2_\perp N +S$$
$$\partial_tN+\left[\Phi,N\right]=-\sigma_\parallel N e^{\Lambda-\Phi}+D\nabla^2_\perp N +S_N$$
$$\partial_tN+\frac{1}{B}\left[\Phi,N\right]+NB\left[\Phi,\frac{1}{B^2}\right]-B\left[NT_e,\frac{1}{B^2}\right]=\sigma_\parallel N \exp\left(\Lambda-\Phi\right)+D\nabla^2_\perp N +S$$

$$\partial_t\left(n_i-n_e\right)+\nabla\left(n_i\vec{u}_i-n_e\vec{u}_e\right)=e\partial_t\left(n_i-n_e\right)+\nabla\left(\vec{j}\right)=0 $$
$$\nabla\left(\vec{j}_{dia}+\vec{j}_{pol}+\vec{j}_\parallel\right)=\nabla_\perp\left(en\left(\vec{u}^{~dia}_i-\vec{u}^{~dia}_e+\vec{u}^{~pol}_i\right)\right)+\nabla_\parallel\left(j_\parallel\right)=0$$
$$\nabla_\perp\left(en\left(\vec{u}^{~dia}_i-\vec{u}^{~dia}_e\right)\right)=B\left[P_e+P_i,\frac{1}{B^2}\right]=B\left(T_i+T_e\right)\left[n,\frac{1}{B^2}\right]$$
$$\left[N,\frac{1}{B^2}\right]=g\partial_yN$$
$$\nabla_\perp\left(qn\vec{u}^{~pol}_i\right)=-\nabla_\perp\left(n\frac{m_i}{B^2}\left(\frac{d\nabla_\perp U}{dt}\right)_\perp\right)$$
$$\left(\frac{d\nabla_\perp U}{dt}\right)_\perp\cong\left(\partial_t-\nu\nabla^2_\perp+\left(\vec{u}_E+\vec{u}_{dia}\right)\nabla_\perp\right)\nabla_\perp U $$
$$\nabla_\perp\left(qn\vec{u}^{~pol}_i\right)=-\nabla_\perp\left(n\frac{m_i}{B^2}\left(\partial_t-\nu\nabla^2_\perp+\frac{1}{B}\left[U,\cdot\right]+\frac{T_i}{qnB}\left[n,\cdot\right]\right)\nabla_\perp U\right) $$
$$\partial_t\nabla^2_\perp \Phi+\left[\Phi,\nabla^2_\perp \Phi\right]=\sigma_\parallel\left(1-\exp\left(\Lambda-\Phi\right)\right)-\frac{g}{N}\partial_yN+\nu\nabla^4_\perp \Phi$$
$$\partial_t\nabla^2_\perp \Phi+\left[\Phi,\nabla^2_\perp \Phi\right]-\frac{B^3}{N}\left[NT_e,\frac{1}{B^2}\right]=\sigma_\parallel\left(1-\exp\left(\Lambda-\Phi\right)\right)-\nu\nabla^4_\perp \Phi$$


$$\partial_tN+\nabla\left(\vec{v_\perp}N\right)=-\sigma_\parallel N\sqrt{T_e} e^{\Lambda-\frac{\Phi}{T_e}}+D\nabla^2_\perp N +S_N$$
$$\partial_t\nabla^2_\perp \Phi+\nabla\left(\vec{v_\perp}\nabla^2_\perp \Phi\right)=\sigma_\parallel \sqrt{T_e}\left(1-e^{\Lambda-\frac{\Phi}{T_e}}\right)-\nu\nabla^4_\perp \Phi$$
$$\frac{3}{2}\partial_tP_e+\frac{3}{2}\nabla\cdot\left(\vec{v_\perp}P_e\right)+P_e\nabla\cdot\vec{v_\perp}=-\sigma_\parallel P_e\sqrt{T_e} e^{\Lambda-\frac{\Phi}{T_e}}+\chi_e\nabla^2P_e+S_{T_e}$$


$$\partial_tN+\frac{1}{B}\left[\Phi,N\right]+NB\left[\Phi,\frac{1}{B^2}\right]-B\left[N,\frac{1}{B^2}\right]=-\sigma_\parallel N e^{\Lambda-\Phi}+D\nabla^2_\perp N+S_N$$
$$\partial_t\nabla^2_\perp \Phi+\frac{1}{B}\left[\Phi,\nabla^2_\perp \Phi\right]-\frac{B^3}{N}\left[N,\frac{1}{B^2}\right]=\sigma_\parallel\left(1-e^{\Lambda-\Phi}\right)+\nu\nabla^4_\perp \Phi$$

$$\partial_tN+\left[\Phi,N\right]=-\sigma_\parallel N e^{\Lambda-\Phi}+D\nabla^2_\perp N +S_N$$
$$\partial_t\nabla^2_\perp \Phi+\left[\Phi,\nabla^2_\perp \Phi\right]=\sigma_\parallel\left(1-e^{\Lambda-\Phi}\right)-\frac{g}{N}\partial_yN+\nu\nabla^4_\perp \Phi$$


$$\alpha\partial^2_xX+\beta\partial_xX+\gamma X=\delta$$

$$\sigma_\parallel \sqrt{T_e}\left(1-e^{\Lambda-\frac{\Phi}{T_e}}\right)=\sigma_\parallel \sqrt{T_e}\left(1-e^{\Lambda-\frac{\Phi^{t}}{T_e}}\right)+\sigma_\parallel \sqrt{T_e}\Phi^{t+1}/T_e$$
	\section{La dérivation des équations} 
	

	$$\begin{cases}
		\displaystyle\frac{\partial N}{\partial t}+\nabla\left(N\cdot v_e\right)=\sigma_\parallel N e^{\left(\Lambda-\Phi\right)}+D\nabla^2_\perp N +S  
		&\text{: N Density}\\
		~\\
		\displaystyle\frac{\partial W}{\partial t}+\nabla\left(W\cdot v_e\right)=\sigma_\parallel \left(1-e^{\left(\Lambda-\Phi\right)}\right)+\frac{g}{N}\partial_y N+\nu\nabla^2_\perp W
		&\text{: W Vorticity} \\
		~\\
		\displaystyle W=\nabla^2_\perp\Phi
		&\text{:~} \Phi \text{~Potential} \\
	\end{cases}$$
	
	$$\begin{cases}
		\displaystyle\partial_t N+\nabla\left(N\cdot v_e\right)=-\sigma_\parallel N e^{\left(\Lambda-\Phi\right)}+D\nabla^2_\perp N +S  
		&\text{: N Density}\\
		~\\
		\displaystyle\partial_t{\nabla^2_\perp\Phi}+\nabla\left(\nabla^2_\perp\Phi\cdot v_e\right)=\sigma_\parallel \left(1-e^{\left(\Lambda-\Phi\right)}\right)-g\partial_y ln(N)+\nu\nabla^4_\perp\Phi
		&\text{: W Vorticity} \\
		~\\
		\displaystyle W=\nabla^2_\perp\Phi
		&\text{:~} \Phi \text{~Potential} \\
	\end{cases}$$


\end{document}
%% MARK: Acknowledgement
\ChapterStar{Remerciements}
\addstarredchapter{Remerciements}
\markboth{Remerciements}{REMERCIEMENTS}
\thispagestyle{preface}
 
Si ces trois années de thèse ont été intenses et intéressantes sur tous
les plans, c'est surtout grâce aux personnes qui m’ont entouré et que je
voudrais ici remercier.

Je tiens tout d'abord à remercier mes deux directeurs de thèse, Gerjan Hagelaar
et Patrick Tamain, pour m'avoir offert la possibilité d'effectuer ces travaux.
Le sujet n'aurait pas pu être mieux choisi, du fluide, de la conception et une
belle problématique! Je leur dois ces trois années
particulièrement agréables et enrichissantes, et leur suis reconnaissant de la
confiance qu'ils m'ont témoignée et de tout ce qu'ils ont pu m'apporter,
humainement et scientifiquement. J'aimerais qu'ils trouvent ici l'expression de toute ma
gratitude et de ma sincère amitié. 

Gerjan, tu as parfaitement encadré cette
thèse ; pour ta vision "pragmatique" des problèmes et ta rigueur
scientifique qui resteront exemplaires pour moi ainsi que pour toutes tes autres
qualités humaines, nogmaals hartelijk dank!
Patrick, merci d'avoir accepté de co-encadrer cette thèse et de m'avoir si bien
accueilli à l'IRFM durant la première année. J'ai grandement apprécié tes
conseils avisés et tout particulièrement tes explications passionnées.
Impétueux dit-on? Bonne chance pour tout ce qui t'attend maintenant!

Alain Simonin, a été l'initiateur de la collaboration entre le GREPHE et
l'IRFM. Il a de plus donné vie à toute une partie de mon sujet de
thèse, et m'a offert l'opportunité de l'accompagner sur les campagnes
expérimentales de Cybele. Sans lui, je ne serais pas en train d'écrire ces mots.
Pour cela je le remercie, ainsi que pour sa gentillesse, sa constante bonne
humeur, et pour avoir accepté de faire partie de mon jury de thèse.

Je suis également très reconnaissant à Pascal Chabert et à Peter Beyer d'avoir
accepté d'être les rapporteurs de cette thèse. Ils ont consacré
beaucoup de temps et d'énergie à l'étude de mon travail et à l'amélioration d'un
manuscrit qu'ils n'ont découvert que tardivement. Qu'ils en soient ici
sincèrement remerciés.

Un grand merci à Claudia Negulescu et à Stanimir Kolev pour leur participation
au jury, et pour m'avoir chiffonner avec leur interrogations légitimes (si
si\ldots).

Au CEA et au LAPLACE j’ai aussi bénéficié d’un excellent
environnement de travail en compagnie de collaborateurs toujours disponibles et
à l'oreille attentive :
Jean-François Artaud, Vincent Basiuk,
Jean-Pierre Boeuf,
Gwenaël Fubiani,
Xavier Garbet,
Laurent Garrigues,
Philippe Ghendrih, 
Laurent Liard,
Yannick Marandet,
Leanne Pitchford,
Yanick Sarazin,
Mireille Schneider. 


Interlocuteurs privilégiés, co-bureaux, co-cigarettes et co-buveurs de café,
merci d'avoir été là pour partager les petites pauses pendant ces longues
journées :
Anne Boron, Jonathan Claustre, Marc Coatanéa, Nicolas Kohen, Grégoire Hornung, Gaël Selig et Shaodong Song.

Merci aussi à tous les autres apprentis ou maintenant ex-apprentis physiciens
qui m'ont accompagné tout au long de cette thèse. Des conférences aux séances de
go et de squash, des petits bars aixois au RU de la fac, vous avez supporté des
discussions parfois bien enflammées : Asma Kallel, Olivier Balosso, Antoine
Merle, Antoine Strugarek,
Romain Baude,
Marc Foletto,
Hugo Bufferand,
Pierre Cotier,
Timothée Nicolas,
Stephanie Panayotis,
Jeremie Abiteboul,
David Zarzoso,
Rémi Dachicourt,
Florent Gauthier,
Miguel Dapena-Febrer,
Rémi Bruno,
François Orain
Philippe Coche,
Yu Zhu.

Pour finir, je voudrais remercier toutes les personnes qui, sans avoir
directement participé à cette thèse, ont, par leur amitié, contribué à sa
réussite : 


Jimmy P. et Julien F. du 16bis, Benjamin
G., Aris J., Alice D., David et Lionel B. de Fontenay.
\\
\mbox{~~~~}Romain A. et Adrien S.
mon trinôme de prépa, les anciens de l'école Thomas G., Wong H., Rémi M., Greg
W., Céline F., Alexandre L., JB G., Minh T., Jonathan C., Antoine C., Aram G., Christophe G., Anthony G., Nhathieu T., Paul B., Tim DV., Adrien P., Rotha C.,
David M., Charles C., \ldots
\\
\mbox{~~~~}Les Aixois Bece, Flo, Imanita, Keut, Pochette,
Ben, Alex, JP, Pascalou, Sandrine. Les Toulousains de la
Maison du Bonheur Arthur, Dario, Justine, Louis, Marc, Mattieu, Seb et
Simon.
\\
\mbox{~~~~}Ma famille, toute la OT, le reste de l'ESIEA, les médecins, les
IBMeurs, les oufs de maisalf et d'ailleurs.
\\
\mbox{~~~~}Julie, qui a écouté avec intérêt mes digressions sur les plasmas,
pour sa tendresse et son sourire.
Mes parents enfin, qui auront toujours été là pour me soutenir et à qui je dédie
cette thèse.




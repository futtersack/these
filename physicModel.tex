\Chapter{Modèles fluides pour le transport magnétisé dans un plasma limité par des
parois}
\chaptermark{Modèles fluides pour le transport magnétisé}
 	\section{Les modèles}
L'étude de phénomènes complexes
 	
Ce chapitre est consacré à la construction du modèle fluide 2D pour les plasmas
froids magnétisés.
La première partie brosse un rapide état de l'art concernant la modélisation
dans le domaine des plasmas froids. Nous présenterons ensuite un modèle de
transport fortement magnétisé, son implémentation numérique (le code TOKAM2D)
utilisé pour caractériser la SOL des tokamaks ainsi que quelques travaux. Enfin
nous détaillerons l'élaboration du modèle hybride plasmas froids/magnétisé à
travers une discussion sur les différentes approches qui ont été abordées.
	
	\section{Plasmas froids, le modèle de dérive-diffusion}
Dans le domaine des plasmas froids, l'approche usuelle de modélisation consiste
à résoudre les équations de dérive-diffusions~\eqref{derivediffusion} pour
chaques espèces
	\section{Plasmas fortement magnétisés}
	 \subsection{Le code TOKAM2D}
Le code TOKAM2D a été initialement développé à l'IMTM\footnote{Institut
Méditerranéen Technologique de Marseille} pour étudier la turbulence
d'interchange qui domine le transport transverse dans le plasma de bord des
tokamaks. Sa modification pour intégrer un forçage de la turbulence par un flux
au lieu d'un gradient d'équilibre a permis de retrouver le caractère
intermittent du transport et la longueur de décroissance du profil de densité
que l'on mesurait expérimentalement. L'étude du modèle a aussi amélioré la
compréhension du mécanisme principal de l'instabilité, dont le déclenchement à
seuil de l'interchange et la nature de type avalanche du transport turbulent.

C'est un code fluide, quasineutre, qui décrit le transport transverse en se
basant sur l'approximation des vitesses de dérive (voir §
\ref{Introduction}-\ref{vitessesDerive}).
		\subsection{Hypothèses du modèle}
Dans les conditions typiques de la SOL, l'utilisation d'un modèle fluide est justifiée par la
forte collisionnalité du plasma. Le libre
parcours moyen $\lambda_{ei}=v_T \nu_{ei}$, de l'ordre du mètre, est très inférieur à la longueur de connexion
parallèle des lignes de champ $L_{||}=2\pi q R\approx 100\text{m}$.
Le plasma est de plus très magnétisé $B\approx$\unit{1}{\tesla}, les fréquences
caractéristiques du transport sont donc très lentes devant la fréquence cyclotronique ionique
 ce qui nous permet de nous placer dans le cadre de l'approche des vitesse de dérives :
\begin{equation}
	\omega\ll\omega_c^i\Leftrightarrow \varepsilon_\omega\equiv\frac{\omega}{\omega_c^i}\ll 1
\end{equation}
D'un autre côté $\rho_L^i$, le rayon de Larmor ionique, d'environ \unit{1}{\milli\meter}, est bien supérieur à la longueur de 
Debye $\lambda_D$, de l'ordre de \unit{10^{-5}}{\meter}, ce qui permet de considérer le plasma quasineutre dans son ensemble
, avec les densités ioniques et électronique équales à $n$. 

On décrit alors le transport du plasma à travers l'évolution de la densité électronique $n_e=n$et du potentiel électrostatique $U$
en résolvant l'équation de continuité des électrons ainsi que l'équation de conservation du courant. Les températures électronique 
$T_e$ et ionique $T_i$ sont supposées constantes, avec un rapport $\tau=T_i/T_e$.
L'hypothèse flûte, qui permet la réduction du problème de trois à deux dimensions fera l'objet de l'une des parties suivantes.
		\subsection{Equations de conservation}
En notant $D_\perp$ le coefficient de diffusion rendant compte des collisions, et $S$ un terme source pour modéliser le flux 
de particules provenant du plasma de coeur, l'équation de continuité des électrons s'écrit :
\begin{equation}
\label{2-ContinuiteElectrons}
	\partial_t n + \nabla\cdot\left(n\mathbf{u}^e\right) = D_\perp\nabla^2_\perp n + S
\end{equation}
Comme discuté dans le chapitre précédent, le flux de matière transverse est essentiellement issu de l'advection par la vitesse de
dérive électrique. L'équation \ref{2-ContinuiteElectrons} se transforme donc en :
\begin{equation}
	\partial_t n + \nabla_{\parallel}\cdot\left(n\mathbf{u}^e_{\parallel}\right) = \frac{1}{B}\left[n,U\right] + D_\perp\nabla^2_\perp n + S
\end{equation}
Le crochet de Poisson $[n,U]=\mathbf{b}.(\nabla_\perp n\times\nabla_\perp U)$ s'obtient en développant le terme 
$\mathbf{u}_E\cdot\nabla_\perp N$
		\subsection{Adimensionnement et réduction du modèle}
		\subsection{Analyse linéaire}
		\subsection{Ajout des conditions aux limites transverses}
		\subsection{Implémentation de l'équation d'énergie}
	\section{Application de l'approche par vitesse de dérive pour les plasmas froids}
		\subsection{Model for highly magnetized low-temperature plasmas}
		\subsection{Dimensionless and reduction of the model??}
		\subsection{Linear analysis??}
		\subsection{Numerical validation}
		\subsection{stability analysis}
		\subsection{stability analysis}
	\section{Elaboration du modèle de plasmas froids magnétisé}
		\subsection{Drift diffusion and magnetized drift diffusion model}
		\subsection{Model for highly magnetized low-temperature plasmas}
		\subsection{Dimensionless and reduction of the model??}
		\subsection{Linear analysis??}
		\subsection{Numerical validation}
		\subsection{stability analysis}
		\subsection{stability analysis}

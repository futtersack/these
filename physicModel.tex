\Chapter{Modèles fluides pour le transport magnétisé dans un plasma en
interaction avec des parois}
\chaptermark{Modèles fluides pour le transport magnétisé}
 	\section{Les modèles}
	\lipsum
 	Un modèle scientifique
 	
	Ce chapitre est consacré à la construction du modèle fluide pour les plasmas froids
	magnétisés. 
	La première partie brosse un rapide état de l'art concernant la modélisation dans le domaine des
	plasmas froids. Nous présenterons ensuite un modèle de transport
	fortement magnétisé et son implémentation numérique (le code TOKAM2D) utilisé pour caractériser 
	la SOL des tokamaks. Enfin nous détaillerons l'élaboration du modèle hybride plasmas 
	froids/magnétisé à travers une discussion sur les différentes approches qui ont été abordées. 
	
	\section{Plasmas froids, le modèle de dérive-diffusion}
	\lipsum{20}
	\section{Plasmas magnétisés de la SOL, l'approche par vitesses de dérive}
	 \subsection{Le code TOKAM2D}
	 Le code TOKAM2D a été initialement développé à l'Institut Méditerranéen Technologique de Marseille 
	 pour étudier la turbulence d'interchange qui domine le tranport transverse dans le plasma de bord des tokamaks. 
	 Sa modification pour permettre un forcage de la turbulence par un flux entrant au lieu d'un gradient moyen 
	 de densité a permit de nombreuses avancées dans la compréhension 
	 des mécanismes de l'instabilité : donnant naissance en outre à une dynamique du transport de style avalanche, 
	 c'est à dire où les relaxations de profils s'effectuent en relachant par intermittence de forts flux
	 de densité. Ces structures auto-organisées de densité se propagent sur de très grandes longueurs, augmentant la
	 largeur de la SOL
		\subsection{Hypothèses, ordering}
		\subsection{Velocities and currents}
		\subsection{Conservation equations}
		\subsection{Dimensionless and reduction of the model}
		\subsection{Linear analysis}
	\section{Elaboration du modèle de plasmas froids magnétisé}
		\subsection{Drift diffusion and magnetized drift diffusion model}
		\subsection{Model for highly magnetized low-temperature plasmas}
		\subsection{Dimensionless and reduction of the model??}
		\subsection{Linear analysis??}
		\subsection{Numerical validation}
		\subsection{stability analysis}
		\subsection{stability analysis}

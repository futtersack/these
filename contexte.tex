
\ChapterStar{Introduction}
\begin{refsection}

\addstarredpart{Introduction}
\markboth{Introduction}{INTRODUCTION}
Comprendre la nature et les mécanismes du transport des particules dans un
plasma en présence d'un champ magnétique est une étape essentielle au
développement et à l'optimisation de nombreuses applications.
Parmis les plus ambitieuses, la fusion thermonucléaire contrôlée, qui
consiste à reproduire sur Terre des réactions de fusion\footnote{La réaction de fusion entre deux atomes d'Hydrogène, 
qui fait naturellement briller les étoiles, est à ce jour la source d'énergie
la plus prometteuse pour soutenir le besoin croissant en énergie de notre civilisation,
économiquement pérenne, techniquement faisable et écologiquement durable}, repose sur un
dispositif de confinement magnétique en forme d'anneau appelé « tokamak ».
La matière, chauffée à des températures supérieures à 150 millions de degrés
Celsius, y est totalement ionisée et doit être confinée par un puissant champ magnétique afin
de rester dans cet état. La réussite de ce projet (d'initiative
internationnale depuis le lancement du programme ITER) dépend alors
fondamentalement de la maîtrise du transport de l'énergie et de la matière au
sein de ce champ magnétique.

Cette problématique, cruciale pour la recherche sur la fusion, est tout aussi
importante dans le développement de sources plasmas froids (partiellement
ionisées et où la température des ions, proche de celle du gaz,
est bien inférieure la température des électrons : $T_e\neq T_i\sim
T_n$) qui opèrent à basse pression, et dans lesquelles l'application d'un champ
magnétique amène à des améliorations critiques pour les performances de la
décharges. Ce type de sources est communément utilisée dans de
nombreux domaines, tels que le traitement et la gravure de surface
\parencite{Lie}, la propulsion spatiale \parencite{Zhu}, ou encore l'injection de neutres pour la
fusion. Le champ magnétique permet alors de limiter la pertes des particules sur
les parois (miroirs magnétiques, magnétrons), de laisser pénétrer dans le
plasma une tension appliquée (propulseurs à effet Hall, extraction
des ions négatifs), et/ou d'obtenir un certain type de chauffage par le couplage
d'énergie entre des ondes électromagnétiques et les électrons (décharges
hélicons, sources à résonnance électron-cyclotron (ECR)). Ces sources ont
typiquement pour paramètres : une densité plasma $n_e\sim\,$10$^16$ -
10$^18\,$m$^{-3}$, une densité de gaz $n_g\sim\,$10$^9\,$m$^{-3}$, une
température électronique $T_e\sim$1-20~eV, une intensité de champ magnétique
pouvant varier de quelques Gauss à 0.1~Tesla, et changent donc de façon notable
par rapport à ceux des plasmas chaud de fusion.

Le développement et l'optimisation de ces sources se basent généralement sur la
combinaison de recherches expérimentales et d'études théoriques, elles-même
fortement secondées par l'introduction d'un nouveau type d'empirisme : la
simulation numérique.
La construction de codes de calculs, traduisant divers niveaux d'abstraction de
modèles physiques, mathématiques et numériques, et donnant accès à une
certaine réalité virtuelle à questionner, peut nous aider à
comprendre et à interpréter les phénomènes et les processus qui apparaissent dans ces
plasmas. Cependant, dans le domaine des plasmas froids, la compréhension de
l'effet du champ magnétique est loin d'être complète, et les méthodes
habituelles de modélisation, basées sur des hypothèses d'ordering parfois
injustifiées, atteignent leur limites dans le cas des sources basse-pression. Ce
problème est devenu particulièrement urgent dans les récents efforts entrepris pour modéliser
la source d'ions négatifs du système d'injection de neutres d'ITER, dans
laquelle le champ magnétique, initialement mis en place pour baisser la
température électronique et faciliter l'extraction des ions négatifs, peut
amener un comportement complexe encore mal compris. 

L'objectif de cette thèse, à travers l'élaboration d'un modèle fluide sans
approximation d'ordering entre les longueurs caractéristiques de ces sources
magnétisées (ie. la dimension du plasma $L$, les libres parcours moyens
$\lambda_{e,i}$ et les rayons de Larmor ioniques et électroniques
$\rho_{Le,i}$),
vise ainsi à améliorer la compréhension et les outils de simulation pour
cette source d'ion, mais avec potentiellement de nombreux bénéfices pour la
modélisation des plasmas froids en général.

\section{État de l'art}


\section{La source d'ions négatifs pour ITER}


\section{Environnement de la thèse}

				
		
		Dans une dernière partie, nous analysons des simulations représentatives de
		plusieurs expérimentations avec différentes configurations de champ en comparant
		les résultats avec des données expérimentales ou issues de modèles développés
		spécifiquement pour certaines applications :
		en géométrie de filtre magnétique pour le propulseur électrostatique PEGASES
		ou la source d'ions négatifs du système à injection de neutre d'ITER, le
		comportement du plasma simulé est conforme aux phénomènes observés avec
		l’apparition d’asymétries dues aux dérives induites par le champ magnétique.
		Dans le cas du champ uniforme de la source d'ions CYBELE le plasma se met en
		rotation autour de l'axe magnétique. L'évolution du plasma et la nature du
		transport est alors déterminée par la compétition entre le transport
		perpendiculaire et parallèle au lignes de champ magnétique. Pour terminer, le
		modèle est appliqué dans le cas fortement magnétisé de la Scrape-Of-Layer des
		tokamaks.
		
\subsubsection{Les colonnes de plasmas}
\parencite{RosenbluthSimon}

\parencite{Sakawa} (Modified Simon Hoh Instability)$E_r\cdot\nabla n > 0$

\parencite{Hoh} Instability Penning discharge

\parencite{Fruchtman} stationnaire

\parencite{Sternberg} plasma infini dans z, symetrie axiale \ldots 

\subsubsection{Les filtres magnétiques}
\parencite{Hemsworth} status of ITER NBI

\parencite{Rosenbluth}\parencite{Chandrasekhar}
%\bibliographystyle{apalike}
%\bibliography{biblio}
\end{refsection}


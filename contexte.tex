
\ChapterStar{Introduction}
\addstarredpart{Introduction}
\markboth{Introduction}{INTRODUCTION}
\begin{refsection}

Cette thèse propose un modèle original pour l'étude de la dynamique et des
		phénomènes de transport qui apparaissent dans les plasmas froids en présence
		d'un champ magnétique. 
		
		Nous adressons la complexité de ce transport à travers l’élaboration d'un
		modèle fluide 2D sans séparation d'échelles et l'analyse de simulations
		représentatives de plusieurs expérimentations.
			
		Si la compréhension du transport magnétisé est essentielle pour maîtriser la
		production d'énergie par fusion thermonucléaire, elle devient aussi d'importance
		grandissante dans la conception et l'optimisation d'applications utilisant des
		sources plasmas basse-pression magnétisées.
		Depuis les premières expérimentations dans les années 1950-60 sur le
		confinement magnétique des plasmas, la question du transport magnétisé a été intensément
		étudiée. Le champ magnétique, qui permet fondamentalement de piéger les
		particules le long de ses lignes de champ, induit dans de nombreuses situations la formation
		d'asymétries et d'inhomogénéités du plasma dans la direction perpendiculaire.
		Celles-ci peuvent en retour conduire à l'apparition d'instabilités et à
		l'augmentation significative du transport transverse, qui passe d'une nature
		collisionnelle, homogène et diffusive à une nature convective non-linéaire et
		dynamique. La complexité de ce transport ne nous est alors accessible qu'à
		travers l'élaboration et l'utilisation de modèles plus ou moins complets.
				
		Le corps de cette thèse L'originalité de notre approche est
		d'étudier le problème du transport magnétisé en tant que tel.
		D'une part nous considérons la dynamique du plasma de manière macroscopique, le
		transport des charges, du courant et de l'énergie étant décrit à travers un jeu
		d'équations fluides, l'hypothèse de quasineutralité et un modèle classique de
		gaine.
		Ensuite nous nous focalisons sur le plan perpendiculaire au champ magnétique.
		C'est en effet dans ce plan que la force magnétique agit, entraînant
		l'apparition des inhomogénéités et des asymétries caractéristiques des plasmas
		magnétisés.
		Enfin, nous conservons à priori l'ensemble des termes du modèle.
		Les approximations et simplifications généralement effectuées sur les
		modèles dépendent d'un ordering d'échelles spatiales et temporelles qui, dans
		les plasmas que l'on cherche à représenter, est variable, voir totalement
		absent.
		La prise en compte de l'inertie dans les équations de transport nous permet de
		plus de capturer la dynamique transitoire du plasma, comme certains types
		d'instabilités. Ce modèle fluide et dynamique peut être utilisé pour explorer
		une large gamme de topologies et d'intensités de champ magnétique.
				
		Dans une dernière partie, nous analysons des simulations représentatives de
		plusieurs expérimentations avec différentes configurations de champ en comparant
		les résultats avec des données expérimentales ou issues de modèles développés
		spécifiquement pour certaines applications :
		en géométrie de filtre magnétique pour le propulseur électrostatique PEGASES
		ou la source d'ions négatifs du système à injection de neutre d'ITER, le
		comportement du plasma simulé est conforme aux phénomènes observés avec
		l’apparition d’asymétries dues aux dérives induites par le champ magnétique.
		Dans le cas du champ uniforme de la source d'ions CYBELE le plasma se met en
		rotation autour de l'axe magnétique. L'évolution du plasma et la nature du
		transport est alors déterminée par la compétition entre le transport
		perpendiculaire et parallèle au lignes de champ magnétique. Pour terminer, le
		modèle est appliqué dans le cas fortement magnétisé de la Scrape-Of-Layer des
		tokamaks.

%\bibliographystyle{apalike}
%\bibliography{biblio}
\end{refsection}



\ChapterStar{Introduction}
\begin{refsection}

\addstarredpart{Introduction}
\markboth{Introduction}{INTRODUCTION}

Le contôle du transport dans la direction
transverse au champ magnétique est un enjeu majeur dans l'optimisation des
futurs réacteurs de fusion. D'une part celui-ci doit être suffisamment
faible pour garantir un bon niveau de confinement de l'énergie : l'apparition de
barrières de transport permettent ainsi des mode de confinement améliorés, plus
propices aux réactions de
fusions~\parencite{GarbetBarriere,Ghendrih}. D'un autre côté, un transport transverse plus important permet d'étaler le dépot de puissance,
typiquement de l'ordre  limiter les contraintes thermiques sur les matériaux en
contact avec le plasma en augmentant la largeur du dépot de puissance, ce qui est alors bénéfique au .

	Les plasmas sont utilisés dans de nombreuses applications industrielles et de
	recherches. Parmis ces Le comportement d'un plasma dans un champ magnétique et
	étudié depuis de nombreuses années dans le cadre de la recherche sur la fusion thermonuclaire par confinement magnétique.  La prise en compte de l'inertie
	dans les équations de transport nous permet de plus de capturer la dynamique transitoire du plasma, comme certains types d'instabilités. Ce modèle fluide et dynamique peut être utilisé pour explorer
		une large gamme de topologies et d'intensités de champ magnétique.
				
		
		Dans une dernière partie, nous analysons des simulations représentatives de
		plusieurs expérimentations avec différentes configurations de champ en comparant
		les résultats avec des données expérimentales ou issues de modèles développés
		spécifiquement pour certaines applications :
		en géométrie de filtre magnétique pour le propulseur électrostatique PEGASES
		ou la source d'ions négatifs du système à injection de neutre d'ITER, le
		comportement du plasma simulé est conforme aux phénomènes observés avec
		l’apparition d’asymétries dues aux dérives induites par le champ magnétique.
		Dans le cas du champ uniforme de la source d'ions CYBELE le plasma se met en
		rotation autour de l'axe magnétique. L'évolution du plasma et la nature du
		transport est alors déterminée par la compétition entre le transport
		perpendiculaire et parallèle au lignes de champ magnétique. Pour terminer, le
		modèle est appliqué dans le cas fortement magnétisé de la Scrape-Of-Layer des
		tokamaks.
		
\subsubsection{Les colonnes de plasmas}
\parencite{RosenbluthSimon}

\parencite{Sakawa} (Modified Simon Hoh Instability)$E_r\cdot\nabla n > 0$

\parencite{Hoh} Instability Penning discharge

\parencite{Fruchtman} stationnaire

\parencite{Sternberg} plasma infini dans z, symetrie axiale \ldots 

\subsubsection{Les filtres magnétiques}
\parencite{Hemsworth} status of ITER NBI

\parencite{Rosenbluth}\parencite{Chandrasekhar}
%\bibliographystyle{apalike}
%\bibliography{biblio}
\end{refsection}


\ChapterStar{Conclusion}
\addstarredchapter{Conclusion}
\markboth{Conclusion}{CONCLUSION}
Dans les sources magnétisées basse-pression actuellement en
développement, le plasma a un comportement complexe que les
approches habituellement utilisées pour modéliser le transport
magnétisé ne permettent pas d'expliquer.

Pour répondre à cette problématique, cette thèse revisite la modélisation
des plasmas froids et propose un nouveau modèle fluide décrivant le transport
magnétisé dans le plan perpendiculaire au champ magnétique.

\begin{center}
\rule{0.6\textwidth}{1pt}
\end{center}

La première année de cette thèse s'est déroulée au CEA de Cadarache.
L'objectif était de se familiariser avec les techniques de modélisation
utilisées pour décrire les plasmas fortement magnétisés de fusion puis d'essayer
de les adapter aux conditions spécifiques des plasmas froids. Cette première
année a donc consisté à :

\begin{itemize}
  \item étudier et utiliser le code TOKAM2D qui simule
  l'instabilité d'interchange dans les plasmas de bord de tokamaks. Le 
  modèle de TOKAM2D est fluide, quasineutre, et décrit l'évolution de la
  densité et du potentiel électrique dans le plan perpendiculaire au champ
  magnétique en se basant sur l'approximation des vitesses de dérive.
  \item inclure la possibilité de définir des conditions aux limites pour
  traiter la présence de parois.
  La version initiale du code était en géométrie bi-périodique pour une résolution des
  équations dans l'espace de Fourier. Il a fallut changer de schéma
  numérique et donc réécrire une grande partie du code, dont une inversion de
  Laplacien.
  \item ajouter la définition d'un champ magnétique
  variable et les termes correspondant dans les équations du modèle.
  \item implémenter une équation d'énergie pour suivre l'évolution de la
  température électronique.
\end{itemize} 
 
 Au cours de cette année, nous avons aussi réalisé une étude sur le courant et
 l'impact des conditions aux limites parallèles et perpendiculaires sur le
 transport. Cette étude montre deux points : d'une part que les conditions aux
 limites sur le courant parallèle influencent fortement le transport et la
 turbulence (prouvant que celles-ci doivent être issues d'une
 théorie bien posée pour capturer la physique du problème). D'autre part elle
 prévient que dans la direction perpendiculaire, en attendant qu'une théorie
 solide se développe, il faut être conscient de l'influence du choix de ces
 conditions aux limites.
	
Une adaptation plus poussée de TOKAM2D pour ajouter l'interaction avec les
neutres a été confrontée à deux problèmes majeurs : 

D'une part, l'hypothèse de
forte magnétisation est discutable pour les ions et s'effondre totalement dans les
zones peu magnétisées des sources de plasma froids ; l'ajout de l'interaction
avec les neutres assure la validité des expressions des vitesses de dérive
quand $B\rightarrow\,$0, mais la fréquence de collision doit être suffisamment
grande, ce qui n'est pas le cas des sources basse-pression\footnote{De plus, l'approche
des vitesses de dérive utilise un développement en ordre de la vitesse
perpendiculaire en supposant que le mouvement des particules se décompose en un
mouvement rapide de moyenne nulle et un lent mouvement de dérive. Même si
l'on peut retrouver un petit paramètre, mixant le rayon de Larmor et le libre
parcours moyen, pour exprimer la vitesse perpendiculaire, le principe de base
de décomposition n'a plus vraiment de sens.}.
Ensuite, le
développement des divergences dans les équations de conservation pose la
question des conditions aux limites à appliquer aux directions perpendiculaires.
 Et plus particulièrement dans l'équation de conservation de la charge pour
 laquelle nous ne connaissons que le courant total entrant ou sortant de la
 gaine. 
   
 Ces travaux sur TOKAM2D ont permis de mettre en avant la nécessité d'inclure
 le terme d'inertie ionique dans les équations de conservation ainsi que les pertes de
 matière, de courant et d'énergie le long de la direction parallèle. En effet
 dans l'équation de la charge, l'inertie ionique a un rôle déterminant à travers
 le courant de polarisation, et les conditions de gaine s'avèrent cruciales
 dans l'évolution du potentiel électrique.
 
 \begin{center}
\rule{0.6\textwidth}{1pt}
\end{center}
 
 La suite de la thèse a été consacrée à l'élaboration d'un
 nouveau modèle et d'un code numérique, MAGNIS, pour étudier le transport
 dans les plasmas froids magnétisés. Les principes de MAGNIS sont les suivants :
 
\begin{itemize}
  \item c'est un modèle fluide, construit sans autre approximation
  d'échelle que celle de la longueur de Debye (quasineutralité).
  \item il décrit le transport d'un plasma dans le
  plan perpendiculaire au champ magnétique. C'est le plan des dérives, des forts
  gradients et des instabilités, essentiel pour comprendre le transport
  magnétisé.
  \item il résout implicitement en temps les équations de conservation de la
  charge et de l'énergie en s'appuyant sur une prédiction-correction des
  vitesses fluides et du flux de chaleur. Chaque espèce ionique a sa propre
  équation de continuité.
  \item  les conditions aux limites parallèles et perpendiculaires sont 
 dérivées de la théorie classique de gaine (critère de Bohm) et donc issues de
 considérations physiques. La vitesse des ions est supposée au minimum
 supersonique en entrée de gaine, mais est calculée de façon auto-cohérente à l'intérieur du plasma.
 \item les opérateurs de divergence sont discrétisés avec une méthode de Volumes
 Finis, et les flux ioniques sont calculés à partir d'un schéma UPWIND ou MUSCL.
 \end{itemize} 

En paramètre d'entrée, on fixe le profil de puissance absordée, la
 pression et les données caractéristiques du gaz ainsi que la géométrie du champ
 magnétique. Les parois peuvent être choisies diélectriques ou conductrices,
 avec la possibilité d'appliquer un bias. En sortie, le code donne l'évolution
 des champs et des courants dans le plasma. MAGNIS est écrit en Fortran,
 et atteint une convergence pour des cas stationnaires en quelques minutes /
 heures suivant les paramètres de la simulation sur une machine standard de bureau.

La construction de MAGNIS a demandé un travail numérique
important pour obtenir une discrétisation des flux
magnétisés minimisant les erreurs d'interpolation (utilisation d'un maillage
dual et pondération entre les vitesses perpendiculaires et les vitesses
croisées). Nous avons aussi implémenté des techniques originales pour augmenter
la stabilité du modèle, comme la relaxation de la vitesse électronique et
l'utilisation d'itérations de Newton pour impliciter certains termes des
équations.

En plus d'un premier test de convergence sur maillage qui a vérifié une
convergence à l'ordre 2 du schéma dans un cas stationnaire, nous avons testé
MAGNIS pour validation sur trois cas simplifiés, chacun représentatif d'une
problématique particulière :

\begin{itemize}
  \item pour la configuration filtre magnétique de PEGASES, nous retrouvons les
  asymétries du plasma, le flux oblique des électrons, le blocage de la dérive
  en haut du filtre magnétique et la loi d'échelle en $1/B$ caractéristique du
  transport magnétisé ; nous avons aussi mis en évidence la probable présence d'ions
  supersoniques, ainsi que l'apparition de phénomènes instationnaires ;
  \item dans l'exemple classique de la colonne magnétisée, conformément
  aux mesures expérimentales, on voit que le
  plasma est instable et en rotation quasi-solide. Des structures de
  densité "en bras" peuvent apparaître et donner une forme de spirale au
  plasma. Dans cette géométrie, le transport du plasma est très sensible à la
  longueur et aux conditions aux limites de la direction parallèle ;
  \item appliqué aux conditions et à la géométrie des plasmas de bord de
  tokamaks, MAGNIS arrive à capturer l'instabilité d'interchange,
  en reproduisant à la fois la dynamique ballistique des avalanches, qui sont
  des flux très intermittents pouvant se propager sur une longue distance, son
  caractère à seuil et la particularité de ballonning de l'instabilité.
  Ce test a aussi montré que le code restait stable confronté à un problème
  de type turbulent.
\end{itemize} 

MAGNIS peut décrire les plasmas des sources basse-pression, en présence et en
l'absence de champ magnétique. L'ajout du terme d'inertie dans l'équation du
courant permet aussi en principe à MAGNIS de capturer les phénomènes
instationnaires et turbulents, caractéristiques du transport magnétisé et encore relativement
inexplorés.

\begin{center}
\rule{0.6\textwidth}{1pt}
\end{center}

MAGNIS a été appliqué à différentes géométries et donne des résultats plus que
satisfaisants dans un grand nombre de situations. Cependant certains points
restent ouverts et les cas à basse pression, quand les phénomènes instationnaires
commencent à apparaître, sont toujours problématiques :

\begin{itemize}

  
  \item MAGNIS reproduit l'instabilité décrite par Simon \& Hoh ainsi
  que l'instabilité d'interchange mais une étude des taux de croissance est
  nécessaire pour valider ces comportements.
  \item pour des filtres magnétiques le modèle prédit des ions supersoniques
  (localement en aval du filtre où $T_e$ est bas) qui ne semblent pas avoir été
  détectés expérimentalement sur PEGASES. Une étude analytique du modèle et des
  comparaisons expérimentales ou avec des codes PIC devront vérifier leur
  existence.
  \item la vérification numérique du modèle est compliquée en présence
  d'instabilités et en dehors des tests de convergence classiques, qui ne sont
  pas toujours applicables, nous manquons d'outils et de techniques pour
  vérifier et caractériser ces phénomènes.
    \item les comparaisons des résultats du modèle avec l'expérimentation du
  projet METRIS sur les profils de densité et
  de courant sont en cours et très encourageantes, mais pour pousser plus
  loin la validation de MAGNIS, des comparaisons systématiques avec les mesures
  expérimentales et des modèles PIC sont maintenant indispensables.
  \item le code a été appliqué sur des cas idéalisés ne comprenant qu'une seule
  espèce ionique. Les méthodes ont atteint maintenant suffisamment
  de maturité pour pouvoir être appliqués à des configurations plus réelles,
  avec plus de chimie et l'ajout d'autre espèces ioniques, dont des ions
  négatifs. Rien n'interdit ce développement dans le modèle, et il ne demandera
  que peu d'ajustement dans le code.

\end{itemize} 

Il y a aussi des questions sur ce que nous pouvons voir avec un code
quasineutre. En supposant la quasineutralité, nous ne pouvons par conséquent
capturer aucun phénomène en lien avec la séparation de charge comme
l'instabilité diochotron et les trous de plasma dans les colonnes magnétisées.
Nous ne décrivons pas non plus la fonction de distribution des ions, nous
supposons $T_i$ uniforme et le tenseur de pression est pris isotrope, ce qui
nous prive d'une partie de la physique cinétique.
Pour l'instant nous avons encore du mal à estimer la validité/les conséquences des
approximations fluides, et il faudra effectuer des comparaisons avec des
modèles PIC pour vérifier l'existence des phénomènes observés avec MAGNIS.





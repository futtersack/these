\ChapterStar{Conclusion}
\addstarredpart{Conclusion}
\markboth{Conclusion}{CONCLUSION}
Les connaissances sur le sujet du transport magnétisé n'ont que très peu évolué
dans le domaine des plasmas froids depuis les années 1960 et ne suffisent plus
pour expliquer le comportement complexe du plasma que l'on peut rencontrer dans
les sources magnétisées basse-pression actuellement en développement.
Ce problème devient particulièrement urgent dans le cadre de la modélisation de
la source d'ions négatifs d'ITER où le plasma, confronté à une barrière
magnétique de l'ordre de la centaine de Gauss, forme une assymétrie prononcée
dans le plan perpendiculaire au champ magnétique. Celle-ci, en rendant
extrêmement inhomogène la densité de courant extraite, réduit considérablement
les performances de la source.

Pour répondre à cette problématique, nous avons développé un nouveau modèle :

\begin{itemize}
  \item basé sur les équations fluides. Une équation de continuité et de vitesse
  fluide pour chaque espèce ionique. Une équation de conservation de la charge
  pour le potentiel. Une équation d'énergie et de flux de chaleur pour les
  électrons.
  \item une équation 
\end{itemize} 

Les modèles fluides développés pour comprendre le transport dans ce type de
source se basent habituellement sur les équations de dérive-diffusion dans leur
forme tensorielle, qui tiennent compte de la forte anisotropie du champ
magnétique.
Cependant, dans des sources basse-pression,
l'approximation standard de forte collisionnalité n'est plus
valide et nous avons vu que la prise en compte de l'inertie des ions devient
essentielle pour décrire correctement le transport. 

La première année de cette thèse, qui a consisté à étudier et comprendre un
modèle décrivant le transport fortement magnétisé dans les plasmas de
bord des tokamaks, a notamment permis de mettre en avant l'importance de ce
terme d'inertie dans l'équation de conservation de la charge, à travers le rôle
du courant de polarisation.
Quand le plasma est très magnétisé, la contribution du courant de polarisation
devient du même ordre de grandeur que celle du courant diamagnétique et ne
peut donc plus être négligée. L'autre point déterminant, qui a conduit à
inclure les pertes le long de la direction parallèle, s'avère crucial dans
l'évolution du potentiel électrique.

MAGNIS a été élaboré en fonction de ces deux observations. Puisque les modèles
habituellement utilisés dépendent d'approximations de
l'équation de la quantité de mouvement et en sont par suite fortement limités,
nous avons choisi de la résoudre entièrement et sans ordering particulier entre les
différentes échelles spatiales du transport. Ainsi, notre modèle est à même de
décrire les plasmas des sources basse pression, en présence et en l'absence de champ magnétique.
L'ajout du terme d'inertie dans l'équation du courant permet aussi à MAGNIS de
décrire les phénomènes instationnaires et turbulent, caractéristiques du
transport magnétisé et encore relativement inexplorés. 

\begin{center}
\rule{0.6\textwidth}{1pt}
\end{center}

Le premier résultat encourageant concerne la facilité avec laquelle le modèle
arrive à décrire la dérive due à la barrière magnétique, là où les autres
modèles rencontraient systématiquement des problèmes de stabilité. Le test de
convergence sur maillage est aussi très concluant, et nous assure de la qualité
des solutions calculées dans un cas simple collisionnel. 

Les trois études de cas apportent quant à eux des éléments de validation
importants pour le modèle, qui arrive à reproduire au moins qualitativement les
comportements déjà observés. L'effet de dérive du flux électronique et la
variation du courant extrait en fonction de l'intensité du champ magnétique sont des
résultats exemplaires bien connus caractéristiques de la géométrie du filtre
magnétique.
Dans la colonne magnétisée, la rotation presque solide du plasma et le
transport des courants (dans les direction perpendiculaires et parallèles) vont
du côté de la théorie classique du transport magnétisé dans le cas d'une dérive
azimutale refermée sur elle-même.
La comparaison avec le modèle de plasma de bord permet quand à elle de nous
assurer que notre description de l'évolution du plasma dans un cas fortement
magnétisé est cohérente en trouvant des résultats très similaires sur les
caractéristiques principales du transport comme la formation des structures
auto-cohérentes et le profil moyen de densité fortement allongé dans la zone
instable du plasma vis-à-vis de l'instabilité d'interchange.

Les cas à plus basse pression, quand les phénomènes instationnaires
commencent à apparaître, sont cependant toujours problématiques. Nous
cherchons encore à expliquer et caractériser les instabilités observées mais
sans savoir réellement comment tester la convergence. Ces phénomènes, qui ne
sont pas systématiquement observés dans les expérimentations, 
persistent cependant à apparaître dans une grande variété de situations, nous
incitant à penser que leur existence est bien réélle. De plus, dans une
géométrie périodique (où la dérive se referme sur elle-même) où des codes PIC
font état de la présence d'une instabilité, MAGNIS reproduit de manière très similaire 
le développement de structures se déplaçant perpendiculairement au champ
magnétique et à son gradient.

\begin{center}
\rule{0.6\textwidth}{1pt}
\end{center}

L'idée commune est que la simulation n'est qu'une illustration de la
modélisation. En nous donnant accès à une réalité virtuelle à questionner et
analyser, les modèles permettent de régler et d'optimiser les applications
concrètes. Cependant un autre point souvent laissé de côté mais tout aussi
intéressant est de voir comment les simulation peuvent aider à améliorer les
modèles. MAGNIS est un bon exemple de cette problématique ; il a été construit
tout au long de cette thèse par des aller-retour continus entre le constat d'une
certaine physique sur les simulations et la modification subséquente du modèle.


Bien que
l'approche fluide ne soit pas fondamentalement justifiée pour décrire la
dynamique des ions dans les plasmas froids, le couplage avec les électrons par
le champ électrique semble améliorer dans un certain sens cette insuffisance en
rendant cohérente l'évolution globale du plasma.

Bien que des phénomènes à la fois dans les
simulations PIC et nos simulations fluides, il est sûr que MAGNIS ne
pourra pas décrire l'ensemble des mécanismes découlant de la cinétique ionique


\ChapterStar{Conclusion}
\addstarredpart{Conclusion}
\markboth{Conclusion}{CONCLUSION}
Les connaissances sur le sujet du transport magnétisé n'ont que très peu évolué
dans le domaine des plasmas froids depuis les années 1960 et ne suffisent plus
pour expliquer le comportement complexe du plasma que l'on
rencontre dans les sources magnétisées basse-pression actuellement en développement.

Pour répondre à cette problématique, cette thèse revisite la modélidation
des plasmas froids et propose un nouveau modèle fluide décrivant le transport
magnétisé dans le plan perpendiculaire au champ magnétique.

\begin{center}
\rule{0.6\textwidth}{1pt}
\end{center}

La première année de cette thèse s'est déroulée au CEA de Cadarache.
L'objectif était de se familiariser avec les techniques de modélisation
utilisées pour décrire les plasmas fortement magnétisés de fusion puis d'essayer
de les adapter aux conditions spécifiques des plasmas froids. J'ai ainsi :

\begin{itemize}
  \item étudié et utilisé le code TOKAM2D qui simule
  l'instabilité d'interchange dans les plasmas de bord de tokamaks. Le 
  modèle de TOKAM2D est fluide, quasineutre, et décrit l'évolution de la
  densité et du potentiel électrique dans le plan perpendiculaire au champ
  magnétique en se basant sur l'approximation des vitesses de dérive ;
  \item inclu la possibilité de définir des conditions aux limites pour traiter
  la présence de parois.
  La version initiale du code était en géométrie bi-périodique pour une résolution des
  équations dans l'espace de Fourier. Il a fallut changer de schéma
  numérique et donc réécrire une grande partie du code, dont une inversion de
  Laplacien ;
  \item ajouté la définition d'un champ magnétique
  variable et les termes correspondant dans les équations du modèle ;
  \item effectué une étude sur le courant et l'impact des conditions aux
  limites parallèles et perpendiculaires sur le transport ;
  \item implémenté une équation d'énergie pour suivre l'évolution de la
  température électronique.
\end{itemize} 

 Une adaptation plus poussée de TOKAM2D a cependant été confrontée à deux
 problèmes majeurs :
 
 \begin{itemize}
   \item l'hypothèse de forte magnétisation est discutable pour les ions
 et s'effondre totalement dans les zones peu magnétisées des sources de plasma
 froids ; l'ajout de l'interaction avec les neutres assure la validité des
 expressions des vitesses de dérive quand $B\rightarrow\,$0, mais la fréquence
 de collision doit être suffisamment grande, ce qui est encore discutable dans
 le cas des sources basse-pression ;
 \item le développement des divergences dans les équations de conservation pose
 la question des conditions aux limites à appliquer aux directions
 perpendiculaires.
 Et plus particulièrement dans l'équation de conservation de la charge pour
 laquelle nous ne connaissons que le courant total entrant ou sortant de la
 gaine. 
 \end{itemize}
   
 Ces travaux ont permis de mettre en avant la nécessité d'inclure
 le terme d'inertie ionique dans les équations de conservation ainsi que les pertes de
 matière, de courant et d'énergie le long de la direction parallèle. En effet
 dans l'équation de la charge, l'inertie ionique a un rôle déterminant à travers
 le courant de polarisation, et les conditions de gaine s'avèrent cruciales
 dans l'évolution du potentiel électrique.
 
 \begin{center}
\rule{0.6\textwidth}{1pt}
\end{center}
 
 La suite de la thèse a été consacrée à l'élaboration d'un
 nouveau modèle et d'un code numérique, MAGNIS, pour étudier le transport
 dans les plasmas froids magnétisés. Quelques mots sur le modèle de
 MAGNIS :
 
\begin{itemize}
  \item c'est un modèle fluide, construit sans autre approximation
  d'échelle que celle de la longueur de Debye (quasineutralité) ;
  \item il décrit le transport d'un plasma dans le
  plan perpendiculaire au champ magnétique. C'est le plan des dérives, des forts
  gradients et des instabilités, essentiel pour comprendre le transport
  magnétisé ;
  \item il résout implicitement en temps les équations de conservation de la
  charge et de l'énergie en s'appuyant sur une prédiction-correction des
  vitesses fluides et du flux de chaleur. Chaque espèce ionique a sa propre
  équation de continuité ;
  \item  les conditions aux limites parallèles et perpendiculaires sont 
 dérivées du critère de Bohm et donc issues de considérations physiques. La
 vitesse des ions est supposée au minimum supersonique en entrée de gaine, mais
 est calculée de façon auto-cohérente à l'intérieur du plasma ;
 \item les opérateurs de divergence sont discrétisés avec une méthode de Volumes
 Finis, et les flux sont calculés à partir d'un schéma UPWIND ou MUSCL ;
 \item en paramètre d'entrée, on fixe le profil de puissance absordée, la
 pression et les données caractéristiques du gaz ainsi que la géométrie du champ
 magnétique.
 Les parois peuvent être choisies diélectriques ou conductrices, avec la possibilité d'appliquer un bias. En
sortie, le code donne l'évolution des champs et des courants dans le plasma ;
 \item MAGNIS est écrit en Fortran,
 et atteint une convergence pour des cas stationnaires en quelques minutes / heures suivant les paramètres de la simulation sur une machine standard de bureau.
\end{itemize} 

La construction de MAGNIS a demandé un travail numérique
important pour obtenir une bonne discrétisation des flux
magnétisés et le développement de techniques originales pour augmenter la
stabilité du modèle.
En plus d'un premier test de convergence sur maillage qui a vérifié une
convergence à l'ordre 2 du schéma dans un cas stationnaire, j'ai testé MAGNIS
pour validation sur trois cas simplifiés, chacun représentatif d'une
problématique particulière :

\begin{itemize}
  \item pour la configuration filtre magnétique de PEGASES, nous retrouvons les
  asymétries du plasma, le flux oblique des électrons, le blocage de la dérive
  en haut du filtre magnétique et la loi d'échelle en $1/B$ caractéristique du
  transport magnétisé ; nous avons aussi mis en évidence la probable présence d'ions
  supersoniques, ainsi que l'apparition de phénomènes instationnaires ;
  \item dans l'exemple classique de la colonne magnétisée, conformément
  aux mesures expérimentales, on voit que le
  plasma est instable et en rotation quasi-solide. Des structures de
  densité "en bras" peuvent apparaître et donner une forme de spirale au
  plasma. Dans cette géométrie, le transport du plasma est très sensible à la
  longueur et aux conditions aux limites de la direction parallèle ;
  \item appliqué aux conditions et à la géométrie des plasmas de bord de
  tokamaks, MAGNIS arrive à capturer la physique de base du transport fortement
  magnétisé, tout en restant suffisamment stable confronté à un problème de type turbulent.
\end{itemize} 

MAGNIS peut décrire les plasmas des sources basse-pression, en présence et en
l'absence de champ magnétique. L'ajout du terme d'inertie dans l'équation du
courant permet aussi à MAGNIS de capturer les phénomènes instationnaires et
turbulents, caractéristiques du transport magnétisé et encore relativement
inexplorés.

\begin{center}
\rule{0.6\textwidth}{1pt}
\end{center}

MAGNIS a été appliqué à différentes géométries et donne des résultats plus que
satisfaisants dans un grand nombre de situations. Cependant certains points
restent ouverts et les cas à basse pression, quand les phénomènes instationnaires
commencent à apparaître, sont toujours problématiques :

\begin{itemize}

  \item la vérification numérique du modèle est compliquée en présence
  d'instabilités et en dehors des tests de convergence classiques (qui ne
  fonctionnent pas toujours) nous manquons d'outils et de techniques pour
  vérifier et caractériser ces phénomènes ;
  \item MAGNIS reproduit l'instabilité décrite par Simon \& Hoh ainsi
  que l'instabilité d'interchange mais une étude des taux de croissance est
  nécessaire pour valider ces comportements ;
  \item dans la configuration filtre, les ions supersoniques n'ont
  jamais été détectés (mais sans jamais avoir été réellement cherchés non
  plus). La réalité de cette instabilité est donc discutable, et doit
  donc être prouvée par exemple avec une étude linéaire du modèle ; 
  \item le code a été appliqué sur des cas idéalisés ne comprenant qu'une seule
  espèce ionique. Les méthodes ont atteint maintenant suffisamment
  de maturité pour pouvoir être appliqués à des configurations plus réelles,
  avec plus de chimie et l'ajout d'autre espèces ioniques, dont des ions
  négatifs. Rien n'interdit ce développement dans le modèle, et il ne demandera
  que peu d'ajustement dans le code ;
  \item les comparaisons avec METRIS sont en cours et très encourageantes,
  mais pour pousser plus loin la validation de MAGNIS, des comparaisons
  systématiques avec les mesures expérimentales et des modèles PIC sont
  maintenant indispensables ;
  \item  il y a aussi des questions sur ce que nous
  pouvons voir avec un code fluide. MAGNIS est un code quasineutre, nous ne
  pouvons par conséquent capturer aucun phénomène en lien avec la séparation
  de charge comme l'instabilité diochotron et les trous de plasma dans les
  colonnes magnétisées ;
  \item nous ne décrivons pas non plus la fonction de distribution des
  ions, nous supposons $T_i$ uniforme et le tenseur de pression est pris
  isotrope. Une grande partie de la physique cinétique nous est donc
  inaccessible ;
\end{itemize} 


\begin{center}
\rule{0.6\textwidth}{1pt}
\end{center}

L'idée commune est que la simulation n'est qu'une illustration de la
modélisation. En nous donnant accès à une réalité virtuelle à questionner et
analyser, les modèles permettent de régler et d'optimiser les applications
concrètes. Cependant un autre point souvent laissé de côté mais tout aussi
intéressant est de voir comment les simulations peuvent aider à améliorer les
modèles. MAGNIS est un bon exemple de cette problématique ; il a été construit
tout au long de cette thèse par des aller-retour continus entre le constat d'une
certaine physique sur les simulations et la modification subséquente du modèle.



\ChapterStar{Résumé}
\addstarredchapter{Résumé}
\thispagestyle{preface}
\markboth{Résumé}{RÉSUMÉ}
Les connaissances sur le sujet du transport magnétisé n'ont que très peu évolué
dans le domaine des plasmas froids depuis les années 1960 et ne suffisent plus
pour expliquer le comportement complexe du plasma que l'on
rencontre dans les sources magnétisées basse-pression actuellement en développement.
Les théories et les méthodes développées pour l'étude du transport magnétisé
dans le cadre de la recherche sur la production d'énergie par fusion
thermonucléaire sont inadaptées pour décrire la dynamique non-ambipolaire de ces
plasmas. En effet, dans ces sources, les ions ne sont que faiblement magnétisés,
les collisions avec les neutres influencent significativement le transport
tandis que les parois, puits à particules omniprésents, contrôlent les profils
d'équilibres.

Pour répondre à cette problématique, cette thèse revisite la modélisation
des plasmas froids et propose un nouveau modèle fluide décrivant le transport
magnétisé dans le plan perpendiculaire au champ magnétique.		
Nous adressons la complexité de ce transport à travers l’élaboration d'un modèle
fluide~2D\textonehalf ~et de son schéma numérique, sans approximation d'ordering
entre les longueurs caractéristiques du plasma magnétisé (i.e. la dimension du
plasma $L$, les libres parcours moyens $\lambda_{i,e}$ et les rayons
de Larmor ioniques et électroniques $\rho_{Li,e}$).
Les équations sont résolues dans le plan perpendiculaire au champ magnétique où
les asymétries et les inhomogénéités représentatives du transport magnétisé
apparaissent, tandis que les conditions aux limites (parallèles et transverses)
sont dérivées de la théorie classique de gaine. La considération de l'inertie
des particules permet de plus de capturer la dynamique transitoire du plasma
ainsi que certains types d'instabilités.
		
Le modèle, supportant une large gamme de topologies et d'intensités de champ
magnétique, est appliqué aux configurations de deux sources d'ions négatifs. 
Les asymétries et inhomogénéités observées expérimentalement sont reproduites
et, dans une géométrie représentant la Scrape-of-Layer des tokamaks, le modèle
est capable de simuler la turbulence d'interchange qui domine le transport
perpendiculaire du plasma de bord.
\newpage
		\thispagestyle{empty}
\cleardoublepage
		\thispagestyle{preface}	
		\ChapterStar{Abstract}
		\markboth{Abstract}{ABSTRACT}
The knowledge on the subject of magnetized transport have remained little changed
in the field of cold plasmas since the 1960s and is no longer sufficient
to explain the complex behavior of the plasma which is
found in low-pressure magnetized sources currently under development.
Theories and methods developed for the study of magnetized transport
in the context of the thermonuclear fusion energy research 
are inadequate to describe the non-ambipolar dynamics of these
plasmas. Indeed, in these sources, ions are weakly magnetized,
collisions with neutral significantly influence the transport whereas
walls are perfect sinks for particles and control equilibrium profiles.

To address this problematic, this thesis revisits the cold plasmas modelling
and proposes a new model describing the magnetized fluid transport
in the plane perpendicular to the magnetic field.
We address the complexity of this transport through the development of a 2D+1/2
fluid model and its numerical scheme without approximation on the ordering
between characteristic lengths scales of magnetized plasmas (i.e. the plasma
size $L$, the mean free path $\lambda_{i,e}$ and Larmor radii of ions and
electrons $\rho_{Li,e}$).
The equations are solved in the plane perpendicular to the magnetic field where
asymmetries and inhomogeneities representative of the magnetized transport
appear, while the boundary conditions (parallel and transverse) are derived
from the classical sheath theory. Moreover, the consideration of particles
inertia makes the model able to capture the plasma transient dynamics and
certain types of instabilities.
	
The model, which supports a wide range of magnetic field strengths and
topologies, is applied to the configurations of two negative ions sources. 
Asymmetries and inhomogeneities observed experimentally are recovered and, in a
geometry representing the Scrape-Of-Layer of tokamaks, the model is able to
simulate the interchange turbulence which is thought to dominate the
perpendicular transport of the edge plasma.
		  
		

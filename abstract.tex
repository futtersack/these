%% MARK: Abstract
\thispagestyle{empty}
\cleardoublepage
\thispagestyle{preface}	
	\section*{Résumé}
	\addcontentsline{toc}{section}{Résumé}
		Cette thèse vise à étudier et à comprendre la dynamique et les phénomènes de
		transport qui apparaissent dans les plasmas froids en présence d'un champ
		magnétique.
		
		La question du transport magnétisé est fondamentale et encore intensément
		étudiée dans le cadre de la recherche sur la fusion thermonucléaire
		contrôlée. Le champ magnétique, en piégeant les particules
		chargées dans des orbites cyclotroniques, permet de confiner le
		plasma dans un volume limité
		
		 Les sources plasma
		basse pression fonctionnant avec un champ magnétique sont largement utilisées dans l'industrie et la recherche,
		cependant la connaissance du transport électronique magnétisé dans ce type
		de plasma non-thermique reste très incomplète. 
		
		de particules, de courant et de température électronique
		Le transport de charges et de
		courant dans ces sources pouvant montrer un comportement complexe et mal compris,
		l'objectif est d'étudier le problème du transport magnétisé en
		tant que tel.
		
		Un nouveau modèle fluide bidimensionnel et dynamique a été élaboré, en
		combinant les méthodes usuelles de modélisation plasma froid avec l'approche
		par vitesse de dérive développée pour la recherche sur les plasmas magnétisés
		de fusion, nous permettant ainsi d'explorer une large gamme de topologies et
		d'intensités de champ magnétique.
		
		Nous analysons des simulations représentatives de plusieurs expérimentations
		avec différentes configurations de champ afin de caractériser le transport
		dans ces plasmas basse température puis nous comparons les résultats avec des
		données expérimentales ou issues de modèles développés spécifiquement pour
		certaines applications. Le comportement du plasma est bien retrouvé pour deux
		différents style de source, avec l’apparition d’asymétries dues aux dérives
		induites par le champ magnétique. 
		
		Le modèle est de plus capable de capturer
		certains types d'instabilités propres à la dynamique transitoire du plasma.		
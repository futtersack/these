\ChapterStar{Résumé}
\addstarredpart{Résumé}
\thispagestyle{preface}
Cette thèse propose un modèle physique et numérique original pour l'étude de la
dynamique et des phénomènes de transport qui apparaissent dans les plasmas
froids en présence d'un champ magnétique.
		
Dans les sources basse-pression magnétisées, telles que la source d'ions
négatifs pour l'injecteur de neutres d'ITER ou le propulseur électrostatique
PEGASES, le comportement du plasma est complexe et mal compris.
Les théories et les méthodes développées pour l'étude du transport magnétisé
dans le cadre de la recherche sur la production d'énergie par fusion
thermonucléaire sont inadaptées pour décrire la dynamique non-ambipolaire de ces
plasmas. En effet, dans ces sources, les ions ne sont que faiblement magnétisés,
les collisions avec les neutres influencent significativement le transport et
les parois, puits à particules omniprésents, contrôlent les profils
d'équilibres.
		
Nous adressons la complexité de ce transport à travers l’élaboration d'un modèle
fluide~2D\textonehalf ~et de son schéma numérique, sans approximation d'ordering
entre les longueurs caractéristiques du plasma magnétisé (ie. la dimension du
plasma $L$, les libres parcours moyens $\lambda_\text{mfp}^{i,e}$ et les rayons
de Larmor ioniques et électroniques $\rho_\text{L}^{i,e}$).
Les équations sont résolues dans le plan perpendiculaire au champ magnétique où
les asymétries et les inhomogénéités représentatives du transport magnétisé
apparaissent, tandis que les conditions aux limites (parallèles et transverses)
sont dérivées de la théorie classique de gaine. La considération de l'inertie
des particules permet de plus de capturer la dynamique transitoire du plasma
ainsi que certains types d'instabilités.
		
Le modèle, supportant une large gamme de topologies et d'intensités de champ
magnétique, est appliqué aux configurations de deux sources d'ions négatifs et
du propulseur PEGASES. Les asymétries et inhomogénéités observées
expérimentalement sont reproduites et, dans une géométrie représentant la
Scrape-of-Layer des tokamaks, le modèle semble capable de simuler la turbulence
d'interchange qui domine le transport perpendiculaire du plasma de bord.
\newpage
		\thispagestyle{empty}
\cleardoublepage
		\thispagestyle{preface}	
		\ChapterStar{Abstract}
This thesis suggests an original physical and numerical model for the study of
the dynamics and transport phenomena that occur in low-temperature plasmas in
the presence of a magnetic field.

In low-pressure magnetized sources , such as the negative ions source of the
ITER neutral injector or Pegasus electrostatique thruster , the plasma behavior
is far from understood. Theories and methods developed to study the magnetized
transport from the research on the production of energy by nuclear fusion are
inadequate to describe the non-ambipolar dynamics of these plasmas.
Indeed in these sources, ions are weakly magnetized, collisions with neutral
significantly affect the global transport and walls, as perfect particles sinks
of particles , control the equilibrium profiles.

We address the complexity of this transport through the development of a 2D+1/2
fluid model and its numerical scheme without approximation on the ordering
between characteristic lengths scales of magnetized plasmas ( ie. the plasma
dimention, the mean free path and Larmor radii of ions and electrons).
The equations are solved in the plane perpendicular to the magnetic field where
asymmetries and inhomogeneities representative of the magnetized transport
appear , while the boundary conditions (parallel and transverse) are derived
from the classical sheath theory. Moreover, the consideration of particles
inertia makes the model able to capture the plasma transient dynamics and
certain types of instabilities.
	
The model, which supports a wide range of magnetic field strengths and
topologies, is applied to the configurations of two negative ions sources and of
the Pegasus thruster. Asymmetries and inhomogeneities observed experimentally
are recovered and, in a geometry representing the Scrape-Of-Layer of tokamaks,
the model seems able to simulate the interchange turbulence which is thought to
dominate the perpendicular transport of the edge plasma.
		  
		

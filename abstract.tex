%% MARK: Abstract
\thispagestyle{empty}
\cleardoublepage
\thispagestyle{preface}	
	\section*{Résumé}
	\addstarredchapter{toc}{section}{Résumé}
		Cette thèse vise à étudier et à comprendre la dynamique du transport
		magnétisé des particules et du courant dans des plasmas basse-température.
		
		Bien que les sources plasma basse pression fonctionnant avec un champ
		magnétique soient largement utilisées dans l'industrie et la recherche, la
		connaissance du transport électronique magnétisé dans ce type de plasma
		non-thermique reste incomplète. Le transport de charges et de courant dans
		ces sources pouvant montrer un comportement complexe et très mal compris,
		l'objectif est d'étudier le problème du transport magnétisé en
		soi. 
		
		Un nouveau modèle fluide bidimensionnel et dynamique a été élaboré, en
		combinant les méthodes usuelles de modélisation plasma froid avec l'approche
		par vitesse de dérive développée pour la recherche sur les plasmas magnétisés
		de fusion, nous permettant ainsi d'explorer une large gamme de topologies et
		d'intensités de champ magnétique.
		
		Nous analysons des simulations représentatives de plusieurs expérimentations
		avec différentes configurations de champ afin de caractériser le transport
		dans ces plasmas basse température puis nous comparons les résultats avec des
		données expérimentales ou issues de modèles développés spécifiquement pour
		certaines applications. Le comportement du plasma est bien retrouvé pour deux
		différents style de source, avec l’apparition d’asymétries dues aux dérives
		induites par le champ magnétique. 
		
		Le modèle est de plus capable de capturer
		certains types d'instabilités propres à la dynamique transitoire du plasma.		